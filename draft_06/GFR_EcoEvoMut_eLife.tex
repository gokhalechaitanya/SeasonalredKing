\documentclass[12pt]{article}
\usepackage{times}
\usepackage[titletoc]{appendix}
\usepackage{graphicx}
\usepackage{lineno}
\usepackage{multirow}
\usepackage[english]{babel}
\usepackage{hyperref}
\hypersetup{
    colorlinks=true,       % false: boxed links; true: colored links
    linkcolor=blue,          % color of internal links (change box color with linkbordercolor)
    citecolor=darkgreen,        % color of links to bibliography
    filecolor=magenta,      % color of file links
    urlcolor= black           % color of external links
}
\usepackage{typearea} 
\usepackage{amssymb}
\usepackage{amsfonts}
\usepackage{amsmath}
\usepackage{enumerate}
\usepackage[round,authoryear]{natbib}

\usepackage{wrapfig}
\usepackage{lscape}
\usepackage{rotating}

\renewcommand{\baselinestretch}{1.2}
\newcommand{\bbar}[1]{\overline{#1}}

\usepackage{color}
	 \definecolor{darkred}{rgb}{0.75,0,0}
	 \definecolor{darkgreen}{rgb}{0,0.5,0}
	 \definecolor{darkblue}{rgb}{0,0,0.75}
	 \definecolor{magenta}{rgb}{0,0,0.75}
\newcommand{\cha}[1]{\textcolor{darkblue}{(#1)}}
\newcommand{\marcus}[1]{\textcolor{darkred}{(#1)}}
\newcommand{\paul}[1]{\textcolor{red}{(#1)}}

\title{\vspace*{-22mm}\bf Eco-evolutionary dynamics of mutualisms}
\author{Chaitanya S. Gokhale$^{1*}$,
Marcus Frean$^{2}$,
 \and Paul B. Rainey$^{1,3}$ \\
\normalsize $^{1}$New Zealand Institute for Advanced Study, Massey University, Auckland, New Zealand, \\
\normalsize $^2$Victoria University of Wellington, Wellington, New Zealand\\
\normalsize $^3$Max Planck Institute for Evolutionary Biology, \\
\normalsize August-Thienemann-Stra{\ss}e 2, 24306 Pl\"{o}n, Germany,\\
}

\date{}

\begin{document}

%\linenumbers
\maketitle

\begin{abstract}
Interactions among members of different species, where both partners profit, are seemingly common in nature. 
Just why they are common is difficult to understand given the advantage to types that take unfairly of partner resources. 
Using a theoretical approach we confirm that interactions between two species defined simply in terms of costs and benefits to each partner are prone to failure. 
However, we show that coexistence becomes possible once intraspecific interactions are incorporated. Of particular significance are interactions within each partner species that generate feedback between intra-and inter-specific dynamics leading to co-existence. 
Incorporation of seasonality and population density generates a general theoretical framework closely connected to empirical data and allows a thorough analysis of conditions for the maintenance of mutualisms.
\end{abstract}

\noindent
Keywords: mutualism,evolutionary game theory,multiple players, population dynamics, seasonality

\tableofcontents

\section{Introduction}
The study of mutualistic relationships - interspecific interactions that benefit both species - is a rich field of experimental, empirical and theoretical study \citep{boucher:book:1985,hinton:PTENHS:1951,wilson:AmNat:1983,bronstein:QRB:1994,poulin:JTB:1995,noe:book:2001,johnstone:ECL:2002,pierce:ARE:2002,kiers:Nature:2003,bergstrom:PNAS:2003,hoeksema:AmNat:2003,bshary:ASB:2004,akcay:PRSB:2007,bshary:Nature:2008,gokhale:PRSB:2012,archetti:JTB:2013} \paul{here are some more: \citep{wardle:Science:2004,jones:PNAS:2004,klein:PRSB:2007,hardy:Science:1975,sachs:TREE:2006}.}

Of central importance are the ecological and evolutionary processes that prevent mutualisms becoming parasitic. 
For example, the existence of mechanisms that allow individuals to reward beneficial interactions while punishing harmful or parasitic behaviour \citep{west:JEB:2002}.
\paul{Also relevant are interactions mediated by species other than those directly engaged in specific mutualisms \citep{palmer:Science:2008}. 
And papers by Bronstein. \citep{visick:JB:2000,warren:GCB:2014,mcfallngai:PLoSB:2014}, pointing to rich and complex sets of ecological interactions that shape the nature of mutualisms.}
Indeed, theoretical studies indicate that in the absence of such higher order interactions, mutualisms are prone to failure \citep{doebeli:PNAS:1998}.

Evolutionary game theory provides a simple and tractable framework for analysing mutualism by a cost-benefit analysis of interactions \citep{trivers:QRB:1971,weibull:book:1995,hofbauer:JMB:1996,hofbauer:book:1998}. 
However, game theoretic approaches require that cost and benefit functions be qualitatively known. This poses empirical challenges given frequent non-linearity in the cost benefit function. From a theoretical perspective, progress can be made using multi-player evolutionary game theory \citep{archetti:EL:2011,archetti:JTB:2013}, but necessary are frameworks that incorporate ecological factors likely to effect the dynamic such as density dependence.
%[These references need to be incorporated: \citep{pierce:BES:1987,noe:TREE:1995,hoelldobler:book:1990,hill:OEC:1989,noe:book:2001,kiers:Nature:2003,stanton:AmNat:2003,stadler:book:2008,kiers:Science:2011}]

Here we develop a framework for the study of mutualisms based on multiplayer evolutionary games and use this to explore the effect of intra-species interactions on the stability of mutualisms.  We also incorporate seasonal and density-dependent effects and show how together these generate intricate and dynamic feedbacks between ecological and evolutionary factors. The rich dynamics observed provide new insights into the asymmetries underpinning mutualisms and reveals the fragile and malleable nature of these delicately balanced interactions.

Beginning with the previously studied interspecies dynamics \citep{gokhale:PRSB:2012} we increase the complexity by including seasonality. We tackle this by varying the duration of the impact of intraspecies and interspecies dynamics. For the complete eco-evolutionary picture to emerge we further include population dynamics. Including such dynamics provides information about the population densities that might be expected to drive the evolution of interactions.  We demonstrate the crucial nature of the feedback between population and evolutionary dynamics that can maintain mutualisms preventing either or both species from going extinct. Mutualism, with a healthy mix of exploiters and cooperators, emerge only when intraspecies interactions are included and persist when ecologically realistic phenomena, like seasonality and population dynamics are incorporated. The rich dynamics provides new insights about the immense asymmetries in mutualisms and the fragility of such delicately balanced interactions.


\section{Model and Results}


\begin{figure}
\begin{center}
\includegraphics[scale=0.5]{Figures/interintra.pdf}
\caption{\small{
\textbf{Evolutionary dynamics with combined inter-intra-species dynamics.}
We assume the interactions between species to be mutualistic described by the snowdrift game \citep{bergstrom:PNAS:2003,souza:JTB:2009,gokhale:PRSB:2012}.
Species $1$ plays a $d_1^{inter}$ player game with species $2$ while species $2$ plays a $d_2^{inter}$ player game.
Each species has two types of players ``Generous" and ``Selfish" who besides interacting with the members of other species, also take part in intraspecies dynamics.
For intraspecies interactions we assume a general framework of synergy and discounting which can recover the \textit{classical} outcomes of evolutionary dynamics \citep{eshel:AmNat:1988,hauert:JTB:2006a,nowak:book:2006}
}
\label{fig:conceptart}
}
\end{center}
\end{figure}


\subsection{Dynamics of interspecies interactions}

With a focus on mutualisms, interspecies dynamics are generated by a multiplayer version of the snowdrift game (also known as the hawk-dove, or chicken game)\citep{bergstrom:PNAS:2003,souza:JTB:2009,gokhale:PRSB:2012} which is discussed in detail in the Supplementary Material (SI). A common benefit is generated by contributions from both species, but there are also costs that may be unevenly shared among the interacting players.

We assume that each species consists of two types of individual: "Generous" $G$ and "Selfish" $S$. If most individuals are “Generous”, then selection will favour selfish types that either with-hold some benefit, or take unfairly of resources from the interacting partner.  However, all individuals in the game lose if there is an insufficient number of “Generous” types.  Hence neither species can be completely "Selfish".  Selection thus favours individuals that act “Selfishly”, while at the same time instigating “Generous” responses from the interacting species. The fitness of each of the types within a species depends on the frequency of “Generous” and “Selfish” types of the other species. Denoting the frequency of the "Generous" types in species $1$ ($G_1$) as $x$, and that in species $2$ ($G_2$) as $y$, the fitness of $G_1$ is given by $f^{inter}_{G_1} (y)$ and that of $G_2$ as $f^{inter}_{G_2} (x)$.

Beginning with multiplayer games rather than the simple two player hawk dove game readily allows non-linearity of interactions to be introduced in the cost-benefit framework with payoffs being on the number of "Generous" (or "Selfish") players. As seen before, this has the potential to demonstrate rich dynamics that are qualitatively different and richer than the dynamics generated by linear interspecies games \citep{bergstrom:PNAS:2003,gokhale:PRSB:2012}.  \paul{Seems that we ought to state at this point the overall outcome of the multiplayer analysis interspecies interactions, even if these have been previously described.  Without such a statement the purpose of this section is not clear. }

\subsection{Dynamics of intraspecies interactions}

For the ensuing analysis of intraspecies dynamics we do not restrict ourselves to any particular interaction structure and thus make use of the general multiplayer evolutionary games framework \citep{gokhale:PNAS:2010,gokhale:DGAA:2014}. Moving from interspecies dynamics where the two types already described are "Generous" and "Selfish". Thus each species is comprised of two different types of individuals. It is possible that a different categorisation exists within a species. Thus if the interactions within a species are say between “Cooperators” and “Defectors”, these types could be made up of a combination of "Generous" and "Selfish" individuals. However for the sake of simplicity we study the dynamics of the interactions between "Generous" and "Selfish" types within a species where the types are defined at the interspecies level. The cost benefit framework described in \cite{eshel:AmNat:1988} and \cite{hauert:JTB:2006a}, makes possible a transition between four classic scenarios of evolutionary dynamics \citep{nowak:Science:2004}. 
For example in our case we can have a dominance of the "Generous” type, or the "Selfish" type or both the types can invade from rare resulting in a co-existence or bistability if both pure strategies are mutually non-invasive. For the intraspecies interactions the fitness of a $G_1$ is then given by $f^{intra}_{G_1} (x)$ and that of $G_2$ is given by $f^{intra}_{G_2} (y)$ and similarly for the "Selfish" types.

Considering only intraspecies dynamics is a common approach employed in evolutionary games. Strategies usually evolve within a population and their fate over time is decided by the standard replicator equation (Hofbauer and Sigmund,1998). In our  case intraspecies dynamics are a part of a bigger picture and hence a fraction of the selective force which will decide the fate of the strategies in a combined fashion along with the interspecies dynamics.

\subsection{Combined dynamics}

Combining both intra and interspecific dynamics provides a complete picture of all possible interactions. While our interest is in mutualism at the level of the interspecies interactions, there are four possible interactions within each species \citep{nowak:Science:2004,hauert:JTB:2006a} (dominance of either type, coexistence or bistability). Since within species interactions for the two different species do not need to be the same, there are in all sixteen different possible combinations. Assuming additivity in the fitnesses of inter and intraspecies fitnesses, the combined fitness of each of the two types in the two species are given by,

%
\begin{align}
	f_{G_1} (x,y) &= p f^{inter}_{G_1} (y) + (1-p) f^{intra}_{G_1} (x) \\
	f_{S_1} (x,y) &= p f^{inter}_{S_1} (y) + (1-p) f^{intra}_{S_1} (x)  \\
	f_{G_2} (x,y) &= p f^{inter}_{G_2} (x) + (1-p) f^{intra}_{G_2} (y) \\
	f_{S_2} (x,y) &= p f^{inter}_{S_2} (x) + (1-p) f^{intra}_{S_2} (y).
\label{eq:combinedfiteqs}
\end{align}
%
The parameter $p$ tunes the impact of each of the interactions on the final fitness that eventually drives the evolutionary dynamics. For $p=1$ the well studied case of the Red King dynamics is recovered \citep{gokhale:PRSB:2012}, while for $p=0$ the dynamics of the two species are decoupled and can be individually studied by the synergy/discounting framework of nonlinear social dilemmas \citep{hauert:JTB:2006a}. 
Of interest here is the continuum described by intermediate values of $p$. However that means we need to track the qualitative dynamics of sixteen possible intraspecies dynamics as $p$ changes gradually from $0$ to $1$ (Appendix \ref{app:combineddyn}). The time evolution of the "Generous" types in both species is then given by,
%
\begin{align}
\dot{x} &= r_x x \left(f_{G_1}(x,y) -  \bar{f}_1(x,y) \right)  \\
\dot{y} &= r_y y \left(f_{G_2}(x,y) -  \bar{f}_2(x,y) \right).
\label{eq:repeq}
\end{align}
%
Rates of evolution are central to co-evolutionary interactions \citep{salathe:TREE:2008}. 
If two species evolve at different rates (here $r_x$ for species $1$ and $r_y$ for species $2$), then it is possible that the balance of benefits becomes skewed \citep{bergstrom:PNAS:2003}. 
Furthermore the balance can also be affected by the number of interacting partners \citep{gokhale:PRSB:2012}. 
This variability in the number of players, provides a powerful method to incorporate a multitude of ecological factors into the analysis of interactions. For example the number of players involved in a game could be different for each interaction: for inter and intraspecies interactions for species $1$ ($d^{inter}_1$, $d^{intra}_1$) and similarly for species 2 ($d^{inter}_2$, $d^{intra}_2$). The interspecies interactions are proxied by the multiplayer snowdrift game which can incorporate threshold effects. For example a certain number of "Generous" cleaner fish may be required to clean their host or a certain number of "Generous" ants maybe required to protect the lyacenid larva from predators. $M_1$ and $M_2$ denote threshold number of such "Generous" individuals required to generate the benefit in the two species. Since the interaction matrices for inter and intraspecies dynamics are completely different, in principle, it is possible to have different costs and benefit functions for the four games (two snowdrift games from the perspective of each species and the intragames within each species).

This generates a diverse and rich set of possible dynamics (SI) which demonstrates the importance of incorporating intraspecific interactions into the study of coevolution between interacting species. For a given set of parameter values the full spectrum of possible dynamics is shown in..., see Figure \ref{fig:appendix}. Even under a large number of assumptions (same parameters such as cost-benefit values, thresholds, number of players etc.) and even where intraspecies dynamics accounts for only $33\%$ ($1-p$) of the cumulative fitness, the qualitative dynamics can be radically different and even result in the stable persistence of the "Selfish" types at intermediate frequencies types (Fig. \ref{fig:mainexampleone}). The traditional mutualistic modeling approach can end up in the fixed points where one of the species can get away with being completely "Selfish" while forcing the other to be "Generous". However no coexistence is possible unless the game changes due to threshold effects \citep{gokhale:PRSB:2012}.


\begin{figure}
\begin{center}
\includegraphics[width=\columnwidth]{Figures/mainexample2.pdf}
\caption{\small{
\textbf{Change in evolutionary dynamics due to inclusion of intraspecies dynamics.} When the fitness of the ``Generous" and ``Selfish" types in both the species is solely determined by the interactions which occur between species (in this case mutualism, $p=1$) then we recover the dynamics as studied previously in \citep{gokhale:PRSB:2012}. The colours represent the initial states which result in an outcome favourable for species $1$ (blue leading to ($S_1$,$G_2$)) and species $2$ (red, leading to ($G_1$,$S_2$)). This can result in the Red King effect and other possible complexities as discussed recently in \citep{gao:SciRep:2015}. However when we start including intraspecies dynamics the picture can be very different.
Even when the impact of intraspecies dynamics is only a $1/3$ on the total fitness of the ``Generous" and ``Selfish" types we see a very qualitatively different picture.
Two fixed points are observed where both the ``Generous" and ``Selfish" types can co-exist in both the species.
All initial states in the interior lead to either one of these fixed points (hence the lack of colours).
However it is still possible to characterise the ``successful" species as one of the equilibrium is favoured by one species than the other.
The horizontal isoclines are for species $1$ while the vertical ones are for species $2$.
The analysis was done for a 5 player game $d_1^{inter} = d_2^{inter} = d_1^{intra} = d_2^{intra} = 5$, $b=2$, $c=1$ and $r_x = r_y /8$ for the interspecies mutualism game while additionally $\tilde{b}_1 = \tilde{b}_2 = 10$ and $\tilde{c}_1 = \tilde{c}_2 = 1$ and $\omega_1 = \omega_2 = 3/4$ for the two intraspecies games within each species. Note that even with symmetric games within each species we can a qualitatively drastic difference when compared to the dynamics excluding intraspecies interactions.  For different intraspecies interactions within each species and for varying $p$ see SI.}
\label{fig:mainexampleone}
}
\end{center}
\end{figure}

\subsection{Effect of seasonality on interaction dynamics}

Mutualisms are some times subject to seasonal effects.  Such seasonal or episodic mutualism run a high risk of phenological partner mismatch as a result of climate change \citep{rafferty:Oikos:2015}. For the evolution of a species this means that the effect of interspecific interaction changes over time.

To analyse such episodic mutualistic events we make use of a time-dependent function $p(t) = (1 + \sin(a t))/2 $ instead of a static variable $p$. For the particular parameter set used in Fig. \ref{fig:mainexampleone} ($p=0.666$ panel), introduction of seasonality maintains the two interior fixed points (they are closer to each other for $p = 0.5$), but this is seen only when oscillations in $p(t)$ are comparable, $a=1$, or faster, $a=10$, [than x] with respect to the evolutionary timescale. For slower oscillations $a=0.1$ cyclic behaviour emerges which is prominent in species $2$ more than in species $1$. Slow oscillations mean that the system spends longer close to the starting value of $p(t)$ and hence the phase in which $p(t)$ starts becomes more and more important for smaller and smaller $a$. This is especially interesting if the stability of the system is qualitatively affected over the $p$ continuum.



\begin{figure}
\begin{center}
\includegraphics[width=\columnwidth]{Figures/oscillating_p_reduced.pdf}
\caption{\small{
\textbf{Seasonal changes in the interspecies interactions affecting the evolutionary dynamics within species.}
We model the impact of the interspecies interaction on the fitness of the different types as in Eqs.~\ref{eq:combinedfiteqs} however instead of a static value for $p$ we introduce seasonality via a simple sine function as $p(t) = (1+\sin(at))/2$.
Here, $a$ denotes how the seasonality time scale relates to the inter-intra-species interactions timescale.
A large $a$ denotes multiple bouts of mutualism affecting fitness for a given evolutionary time step while a small $a$ denotes fewer of such bouts within the same evolutionary time step.
The trajectories shown in the panels are obtained by numerical interactions with initial conditions $x = y = \{0.1,0.9\}$ and a step size of $\Delta x = \Delta y = 0.1$.
The background colour is obtained by a finer grain of $\Delta x = \Delta y = 0.01$ and depict the same outcomes as in Fig. \ref{fig:mainexampleone}, with gray representing the outcome that none of the edge equilibria are reached.
For comparable or larger $a$ the dynamics under oscillations can be captured by the average dynamics (at $p = 0.5$) however for small $a$ we a see qualitatively different outcome.
Furthermore the phase in which the oscillating function begins is more important for smaller and smaller $a$ especially if the stability of the fixed points changes as $p$ changes (see Fig. \ref{fig:appendix} panel (b) x (b) across the $p$ continuum).
 }
\label{fig:oscillations}
}
\end{center}
\end{figure}


\subsection{Effect of population density on interactions dynamics}

To this point we have considered each species to consist of two types of individuals. However population size can change over time. Assuming that ecological changes are fast enough that they can be averaged, it is usually possible to ignore their effect on the evolutionary dynamics. It is now possible to show that evolution can happen at fast timescales, comparable to those of the ecological dynamics \citep{post:PTRSB:2009,beaumont:Nature:2009,hanski:PNAS:2011,sanchez:PLoSB:2013}. Hence it is necessary to consider not just evolutionary dynamics but eco-evolutionary dynamics together.

%
\begin{figure}
\begin{center}
\includegraphics[scale=0.5]{Figures/popdyninterintra.pdf}
\caption{\small{
\textbf{Population and evolutionary dynamics with combined inter-intra-species dynamics.}
As with the interactions described in Fig. \ref{fig:conceptart} the two species consist of two types of individuals ``Generous" and ``Selfish".
Since the two species can in principle occupy different environmental niches, they  can have non-overlapping population carrying capacities.
The normalised carrying capacity in both species is $1$ and we have $x_1 + x_2 + z_1 = 1$ (for species $1$) where $x_1$ and $x_2$ are the densities of the ``Generous" and ``Selfish" types respectively (similarly with $y$ and $z_2$ in species $2$). 
The parameter $z_1$ represents the remaining space into which the population can still expand into.
For $z_1 = 0$ the species $1$ is at its carrying capacity while for $z_1 = 0$ it is extinct.}
\label{fig:conceptartpopdyn}
}
\end{center}
\end{figure}
%

To include population dynamics in the previously considered scenario, we reinterpret $x_1$ as the fraction of "Generous" types and $x_2$ as the fraction of "Selfish" types in species $1$. Further we have $z_1 = 1 - x_1 - x_2$ as the empty spaces in the niche occupied by species $1$. Similarly we have $y_1$, $y_2$ and $z_2$ (Fig.~\ref{fig:conceptartpopdyn}). This approach has previously been explored in terms of social dilemmas in \citep{hauert:PRSB:2006}. 
We adapt and modify it for two species and hence now the dynamics of this complete system is determined by the following set of differential equations,
%
\begin{align}
	\dot{x_1} &= r_x x_1 (z_1 f_{G_1} - e_1)  \\
	\dot{x_2} &= r_x x_2 (z_1 f_{S_1} - e_1) \\
	\dot{z_1} &= - \dot{x_1} - \dot{x_2} 
\end{align}
%
for species 1, and
%
\begin{align}
	\dot{y_1} &= r_y y_1 (z_2 f_{G_2} - e_2)  \\
	\dot{y_2} &= r_y y_2 (z_2 f_{S_2} - e_2) \\
	\dot{z_2} &= - \dot{y_1} - \dot{y_2} 
\end{align}
%
for species 2.  $e_1$ and $e_2$ is the death rates of the two species. Setting $e_1 = \frac{z_1 (x_1 f_{x_1} + x_2 f_{x_2}) }{x_1 + x_2}$ and $e_2 = \frac{z_2 (y_1 f_{G_2} + y_2 f_{S_2}) }{y_1 + y_2}$ we recover the two species replicator dynamics as in Eqs.~\ref{eq:repeq} (For the sake of brevity we avoid showing the fitnesses in their the functional forms). In this setup however the fitnesses need to be re-evaluated as now we need to account for the presence of empty spaces (See SI).

The dynamics can be simplified by focusing on the proportion of "Generous" types in both the species thus $g_1 = x_1/(1-z_1)$ and $g_2 = y_1/(1-z_2)$ whose time evolution is given by,

\begin{align}
	\dot{g_1} &= r_x z_1 g_1 (1-g_1) (f_{G_1} - f_{S_1})  \\
	\dot{z_1} &= e_1 (1-z_1) - r_x z_1 (1-z_1) (g_1 f_{G_1} -  (1-g_1) f_{S_1})
\end{align}
and
\begin{align}
	\dot{g_2} &= r_y z_2 g_2 (1-g_2) (f_{G_2} - f_{S_2})  \\
	\dot{z_2} &= e_2 (1-z_2) - r_y z_2 (1-z_2) (g_2 f_{G_2} -  (1-g_2) f_{S_2})
\end{align}
%
where everywhere we have $x_1 = g_1 (1-z_1)$ (with $x_2 = (1-g_1) (1-z_1)$) and $y_1 = g_2 (1-z_2)$ (with $y_2 = (1-g_2) (1-z_2)$) in the fitnesses as well.

Interactions at varying population densities affect group size formation which now includes the possibilities of player positions being left empty. Thus for smaller population densities the interaction groups are small and vice versa at higher densities. The effect of group size on evolutionary dynamics is well documented and can change the results qualitatively \citep{pacheco:PRSB:2009,souza:JTB:2009}. 
Such a two species multi-type interaction system is a complicated but likely realistic depiction of most the mutualisms observed in nature. However given this complexity, we need to look at the dynamics within the two species simultaneously.

We take the most stable situation observed in the dynamics when population dynamics is absent (Fig.~\ref{fig:mainexampleone}) which shows two internal stable equilibria and now incorporate population dynamics. The results are summarised in Figure \ref{fig:popdyn} where we plot the evolutionary parameter (fraction of "Generous" in each species) against the ecological parameter, the population density (or rather in this case the empty spaces). Thus the populations can now go extinct, in a deterministic sense, when the trajectories lead to the population density going to zero (red region).  Since in principle there will be four values changing dynamically making the full dynamics too complex to visualize in two dimensions. Hence we fix the initial abundance of "Generous" individuals and the density in one of the species while observing the dynamics in the other. In the left panel the initial abundance of the “Generous” types in species 2 is 0.5 starting at a density of 0.5 as well. Starting at all possible initial condition in species $1$ in the phase space of ecology (density) and evolution (abundance) we see the presence of a single internal equilibrium and two other interesting extreme equilibria. One with full “Generosity” at an intermediate density and another which leads to extinction with no generous types. Equilibrium selection at a between population level can thus occur between the completely monomorphic population equilibrium and the ones reaching the internal equilibrium Alternatively starting at an initial abundance and density of $0.6$ in species 1 and all initial conditions in species $2$ we see the presence of two internal fixed points. The trajectories separating the attraction domains of the two internal equilibria can intersect in the figure as the actual basins of attractions (and hence the separatrices) lie in a higher four dimensional space. The presence of two internal equilibria can allow for the process of equilibrium selection to occur between multiple populations of heterogeneous nature.

\begin{figure}
\begin{center}
\includegraphics[width=\columnwidth]{Figures/mainexamplepopdyn2.pdf}
\caption{\small{
\textbf{Dynamics of evolutionary strategies and population density for an intraspecies coexistence game with interspecies mutualism.}
With exactly the same parameters as that of Figure \ref{fig:mainexampleone} with  symmetric death rates $e_1 = e_2 = 0.05$ we show two different numerically evaluated examples.
Left Panel: shows the outcomes in species $1$ when starting from $0.5$ fraction of ``Generous" individuals in species $2$ at half carrying capacity $z_2 = 0.5$.
While most of the initial conditions lead to an extinction of species $1$ (red), there exists a fixed point which can be reached when most of species $1$ is ``Generous" and close to carrying capacity (green). For the same or higher fraction of $G_1$ but lower population density, species $1$ can end up being completely ``Generous" (blue).
Right Panel: shows the outcomes in species $2$ when starting from $0.6$ fraction of ``Generous" individuals in species $1$ with empty spaces proportion of $z_1 = 0.6$.
When species $2$ is mostly made up of ``Selfish" types then it leads to species extinction (red), For intermediate levels of ``Generous"individuals there exists an internal equilibrium (dark green). However another stable equilibrium exists as well as even higher densities of ``Generous" types closer to full carrying capacity (green).
Equilibrium selection is thus possible for species $2$ in this case where it is preferable to have an intermediate number of ``Selfish" types.
\label{fig:popdyn}
}
}
\end{center}
\end{figure}



\section{Discussion}

Interspecies interactions, such as mutualism (or antagonist relationships as in predator-prey) are typically considered in the absence of within species interactions. The converse is true when intraspecies interactions are of interest. The major body of work focusing on within population social dilemmas between "Cooperators" and "Defectors" is an example of the same. This simplification is useful when distilling interactions at different levels of community scale. However when inter and intraspecies interactions are interdependent then the feedbacks between the two levels cannot be ignored \citep{schluter:PlosB:2012}.

In principle the methods developed here are capable of handling a diverse array of inter and intraspecific interactions. For interspecific interactions our focus is on mutualisms. Mutualistic interactions between two species can be represented by a bimatrix game. The components of each of the two game matrices need not be correlated as long as they independently satisfy the inequalities leading to a Snowdrift game. By including a greater degree of biological realism including intraspecies interactions, population dynamics and seasonality we have shown that mutualistic interactions can be underpinned by cooperative as well as exploitative strategies is possible in mutualism. A fragile balance of parameters thus appears to underpin the maintenance of mutualisms. If within each species "Generous" and "Selfish" interactions result in coexistence, then this can outweigh the competition they experience at the interspecies level. Note however that at the interspecies level the competition of a "Generous" individual is with "Selfish" individuals from the other species. While the "Selfish" individuals from the other species can drive "Generous" individuals within a species extinct, co-existence between "Generous" and "Selfish" within the same species can overcome the pressure for extinction. In this way, mutualism can be maintained but it comes at a cost of also maintaining a significant level of exploiters. Interactions governing the intraspecies dynamics crucially affect the qualitative outcome seen overall in the success of a species in interspecific interactions. In fact the coevolutionary dynamics between two species is determined together by the inter as well as the intraspecific interactions.

A fragile balance of parameters appear to maintain mutualisms. If within each species the "Generous" and "Selfish" interactions result in coexistence then it can outweigh the competition experienced at the interspecies level. Note, however, that at the interspecies level the interaction of a "Generous" individual is with "Selfish" individuals of the other species. While the "Selfish" individuals from the other species can drive "Generous" individuals within a species extinct, co-existence between "Generous" and "Selfish" within the same species can overcome the pressure for extinction. In this way, mutualisms can be maintained but at a cost of also maintaining a significant level of exploiters. Interactions governing intraspecies dynamics crucially affect the qualitative outcome seen overall in the success of a species in interspecific interactions. In fact the coevolutionary dynamics between two species is determined together by the inter as well as the intraspecific interactions.

It is natural to imagine that the observed mutualism may be seasonal and the interactions are not a continuous feature of the evolutionary trajectory of a species. While the simple case makes predictions possible, inclusion of seasonality inserts a time-dependent factor which makes analytical reasoning difficult. However given the patterns of episodic interactions and studies of mutualistic relationship obtained from field studies over the decades it might be possible to include the seasonal component in future analysis on realistic systems to see how interactions change due to drastic climate change. Including this feature informs as to the dependence of a given mutualism on environmental factors.

While tropical species, such as the various varieties of fig (Ficus), can flower all year round, their mutualistic relationships (with wasps) runs a lower risk of being destabilised as the opportunity for forming an association is available all year round. In contrast, mutualisms are at a high risk of being destabilised if they are episodic. The interactions have to coordinate in time and space as well as in densities.  In ant-aphid mutualisms, the number of attending ants increases up until mid summer, but thereafter declines resulting in rapid extinction in the absence of attending ants \citep{yao:Oikos:2000,yao:JIS:2009}. 
Similarly, bird pollinators of numerous plants are sensitive to environmental and ecological changes which can occur naturally or catalysed by anthropogenic activity. The difference in the timescales of the evolutionary process and environmental fluctuations highlights the fact that averaging out the environmental effects might not always be possible. The system can show qualitatively different behaviour from the average dynamics depending on the kind of interactions initially involved within and between species.

An ecologically important example of a species specific mutualism is that of sunbirds, particularly the Malachite Sunbird (\textit{Nectarine famosa}) and the geophyte \textit{Brunsvigia littoralis}. Besides species specific mutualisms being sensitive to the environmental variation \citep{aizen:Science:2012}, \textit{B. littoralis} populations occur often in low densities \citep{geerts:SAJB:2012} and threatened by rapid urbanisation. A good example of biological interference is the obligate mutualism between Acacia (e.g. bullhorn acacia \textit{A. cornigera}, but also other species) and the acacia ant (\textit{Pseudomyrmex ferruginea}) where browsers have been implicated to have been one of the major selective forces for this mutualism \citep{brown:Ecology:1960,janzen:Evolution:1966}. 
An example with a stronger economic impact, connected to humans comes from the honeybee, \textit{Apis mellifera}. 
A species of immense capital importance, the colony collapse of this pollinator has been attributed to numerous causes ranging from the pesticides to biological interference from parasites and pathogens as well as a change in the environment \citep{nazzi:PLosPath:2012}.

The framework presented herein incorporates exactly these essential elements of interspecies interactions and changing environments (biotic or abiotic), and predicts a significant impact of population dynamics of the interactors. While our focus is currently on mutualism, it is easy to change the interactions by modifying the game defining the interspecies interactions. Including population dynamics and the real threat of extinction needs to be acknowledged when modelling such scenarios. Only then can better conservation tactics be formulated which are not solely based on evolutionary predictions but eco-evolutionary dynamics.

With attention on one of the most well studied examples of mutualism, the squid-vibrio symbiosis, it is hard to exclude population dynamics at least from one of the interacting species \citep{nyholm:NRM:2004}. The diel pattern of the host squid is associated with oscillations in the population density of the symbiont \textit{Vibrio fischerei}. Since the growth rates of the two species differ vastly, the population size turnover inside the squid needs to be managed. While the squid makes use of the full light organ at night camouflaging itself from predators, at dawn it expels almost 95\% of the bacteria. The squid lies buried underground during the day and the remaining 5\% of the bacteria repopulate the light organ again reaching saturation by mid-afternoon. While in our model the population dynamics of the mutualists are driven by the empty spaces in their own niches, this particular example beckons a specific modeling approach where the host itself acts as the niche environment of the symbiont and controls the population density too. Since such density-dependent dynamics can determine the viability or extinction of a species, it is important to incorporate such factors in the analsyis of mutualisms.  The framework developed here is particularly suited to this purpose. It aids in elucidation of interactions possibly responsible in generating the dynamics which we observe in nature and provide criteria under which the observed dynamics are being maintained and ways to explore their stability under varying crucial parameters.

‘A mutualist today may be a parasite of the mutualism tomorrow’ \citep{janzen:bookchapter:1985}. 
This study draws attention to the dynamic nature of mutualisms and the sliver of parameter space over which they are maintained. Slight changes in numerous ecological and evolutionary parameters can tip the balance leading to situations where one mutualist becomes exploited by its partner species leading to extinction of one or both partners. While our discussion has focussed on mutualisms, the framework developed here can be readily extended to the study of a broad range of interactions.  For example, opportunistic pathogens like \textit{H. influenzae} colonise their human hosts without — for the most part — causing harm.  How this interaction changes and its dependence on the environment (in this case the presence/absence of cofactors) could be a possible extension.

\textbf{Acknowledgements}. \cha{CSG acknowledges funding from the New Zealand Institute for Advanced Study and time spent at Victoria University of Wellington. \ldots }


\bibliographystyle{plainnat}

\bibliography{\string~/Bibtex/et.bib}


\renewcommand{\theequation}{A.\arabic{equation}}
\setcounter{equation}{0}

\renewcommand{\thefigure}{A.\arabic{figure}}
\setcounter{figure}{0}

\begin{appendices}

\section{Interspecies Evolutionary Dynamics}

Traditional coevolutionary models consider interspecific dependence only \citep{roughgarden:TPB:1976,roughgarden:book:1983}.
Since in our case each the interactions between the species are mutualistic and each species consists of two types of individuals ``Generous" and ``Selfish", the following Snowdrift game is an appropriate representation of the interactions.


\subsection*{The snowdrift game}
\label{appA}
\subsubsection*{Two player setting}
Two drivers are stuck in a snowdrift.
They must shovel away the snow (paying the cost $c$) to reach home (benefit $b$) but there are three possible outcomes to this scenario.
One of the driver shovels while the other stays warm in the care ($b-c$ and $b$), both the drivers share the workload and shovel away the snow ($b-c/2$ and $b-c/2$) or none of them gets out of the care and they both remain stuck ($0$ and $0$).

Putting this game in perspective of the two species (i.e. the two drivers represent the two different species) we get the matrix,\\
%
\begin{equation}\label{}
\begin{array}{cc|cc}
\multicolumn{4}{l}{\textit{Species 1 payoff:}} \\
\hline\hline
& & \multicolumn{2}{c}{\text{Species 2}}\\
&	&	G_2		&	S_2	\\
\hline
 \multirow{2}{*}{Species 1} & G_1 	& b-c/2 &	b-c \\
&	S_1	&  b & 0 \\
 \hline\hline
\end{array}
\hspace{1cm}
\begin{array}{cc|cc}
\multicolumn{4}{l}{\textit{Species 2 payoff:}} \\
\hline\hline
& & \multicolumn{2}{c}{\text{Species 1}}\\
&	&	G_1		&	S_1	\\
\hline
 \multirow{2}{*}{Species 2} & G_2 	& b-c/2 &	b-c \\
&	S_2	& b & 0  \\
 \hline\hline
\end{array}
\end{equation}
%
where strategy $G$ stands for being \textit{``Generous"} and shoveling the snow while $S$ stands for being \textit{``Selfish"} and just sitting in the car.
For $b=2$ and $c=1$ we recover the matrix used in \citep{bergstrom:PNAS:2003}.

For the snowdrift game in a single population for which the pairings are formed at random, there exists a single, stable internal equilibrium.
Hence the population will evolve to a polymorphism which is a combination of \textit{``Generous"} and \textit{``Selfish"} individuals.
But in a two species system (pairs still random, but one from each species), this stable equilibrium turns into a saddle point: a small deviation from this fixed point leads the system to one of the stable fixed point where one of the species is completely \textit{``Generous"} and the other one is completely \textit{``Selfish"}.

\subsection*{Multiplayer setting}


Following Souza et al. \citep{souza:JTB:2009},  
a multiplayer snowdrift game can be described by the payoff entries
\begin{eqnarray}
\Pi_{G_1} (k)  &=& \begin{cases} b-\frac{c}{k} & \textrm{if } k \geq M \\  -\frac{c}{M} & \textrm{if } k < M \end{cases} 
\\
\Pi_{S_1} (k)  &=& \begin{cases} b & \textrm{if } k \geq M \\ 0 & \textrm{if } k < M. \end{cases}
\label{eqintergamepayoffs}
\end{eqnarray}
%
All players get the benefit $b$ if the number of generous individuals in both species combined, $k$, is greater than or equal to the threshold $M$.
For the generous individuals, their effort is subtracted from the payoffs.
The effort is shared if the quorum size is met ($\frac{c}{M}$), but is in vain for $k<M$. 
For two player games we had $k=1$ but multiplayer games provide the possibility of exploring this threshold aspect of collective action games.
From these payoff entries we need to calculate the average fitnesses.
For simplicity we just illustrate the fitnesses of the strategies in species $1$.
For a $d_1^{inter}$ player game for species $1$ we need to pick $d_1^{inter}-1$ other individuals from species $2$.
Assuming random sampling the composition of the formed groups is given by a binomial distribution.
Summing over all possible compositions of groups we arrive at  the average fitnesses of the two strategies in species $1$,
%
\begin{align}
f^{inter}_{G_1} (y) &= \sum_{k=0}^{d_1^{inter} -1} \binom{d_1^{inter} -1}{k}y^k (1-y)^{d_1^{inter} -1-k} \Pi_{G_1}(k+1)  \\
f^{inter}_{S_1} (y) &= \sum_{k=0}^{d_1^{inter} -1} \binom{d_1^{inter} -1}{k}y^k (1-y)^{d_1^{inter} -1-k} \Pi_{S_1}(k),
\label{interfiteqs}
\end{align}
%
and similarly $f_{G_2}^inter$ and $f_{S_2}^inter$ for species $2$.

Note that here for the sake of notation we have assumed the same values of benefits and costs, thresholds for the two species. However along with the number of player $d_1^{inter}$ and $d_2^{inter}$, these parameters could be very well different for the two species.
For asymmetric bi-matrix games there is a difference in the dynamics between the standard replicator dynamics and the alternative dynamics put forward by Maynard-Smith \citep{maynard-smith:book:1982}.
In this case the replicator equations cannot be simplified by removing the average fitness from the denominator and can give rise to qualitatively different dynamics. 
Then one has to resort to difference rather than differential equations.

\section{Intraspecies Evolutionary Dynamics}
\label{appB}

For elucidating the intraspecies dynamics we will focus on species $1$ as the analysis is analogous for species $2$.
Within species dynamics can in principle be completely different from the between species interactions. 
We can have a multistrategy multiplayer game within a species but to keep things simple we assume a two strategy multiplayer game.
The partitioning of the individuals into two strategies follows the same partitioning as in the inter species interactions as of ``Generous" and ``Selfish". 
In principle we could have two different labels for the strategies in the intraspecies interactions and the ``Generous" and ``Selfish" categories could be split into them.
However for the sake of simplicity we assume the same categorisation as at the inter species level.

\subsection*{Synergy/Discounting Framework}
We model the within species interactions by making use of a general framework of costs and non-linear benefits \citep{eshel:AmNat:1988,hauert:JTB:2006a} which can potentially encompass all different types of (traditionally studied) social interaction structures qualitatively \citep{nowak:book:2006}, i.e., dominance of either type, coexistence and bistability.
Since the categorisation of the strategies at the intraspecies level is the same as that of the inter species level, for species $1$, $x$ and $1-x$, are the frequencies of ``Generous" and ``Selfish" type. 
The ``Generous" and ``Selfish" in species $1$ play a $d_1^{intra}$ player game.
Thus the fitnesses of of the two types are defined as \citep{hauert:JTB:2006a},
%
\begin{align}
	f^{intra}_{G_1} (x) &= \sum_{k=0}^{d_1^{intra} -1} \binom{d_1^{intra} -1}{k}x^k (1-x)^{d_1^{intra} -1-k} \Gamma_{G_1}(k+1)  \\
	f^{intra}_{S_1} (x) &= \sum_{k=0}^{d_1^{intra} -1} \binom{d_1^{intra} -1}{k}x^k (1-x)^{d_1^{intra} -1-k} \Gamma_{S_1}(k).
\label{intrafiteqs}
\end{align}
%
where the payoffs are given by,
\begin{align}
	\Gamma_{S_1} (k) = \frac{\tilde{b}}{d_1^{intra}} \sum_{i=0}^{k-1} \omega^i  \\
	\Gamma_{G_1} (k) = \Gamma_{S_1} (k) - \tilde{c}.
\label{eqintragamepayoffs}
\end{align}
%
Thus the ``Selfish" get a fraction of the benefit which is scaled by the factor $\omega$, which determines whether the benefits are linearly accumulating ($\omega=1$) for increasing number of ``Generous" individuals, synergistically enhanced ($\omega>1$) or saturating ($\omega<1$).
Note that the costs and benefits in the within species game need not be the same as in between species ($b\neq \tilde{b}$ and $c \neq \tilde{c}$).


\section{Combined Evolutionary Dynamics}
\label{app:combineddyn}

The average payoffs are then assumed to be a linear combination of the interspecies and intraspecies interactions where the parameter $p$ determines the strength of each of the interactions such that,
%
\begin{align}
	f_{G_1} (x,y) &= p f^{inter}_{G_1} (y) + (1-p) f^{intra}_{G_1} (x)  \\
	f_{S_1} (x,y) &= p f^{inter}_{S_1} (y) + (1-p) f^{intra}_{S_1} (x).
\label{fiteqs}
\end{align}
%
Following the same procedure for the two strategies in species $2$ leads to the average fitness
%
\begin{align}
\bar{f}_1 (x,y) &= x f_{G_1} (x,y)+(1-x) f_ {S_1}(x,y)  \\
\bar{f}_2 (x,y) &= y f_{G_2} (x,y)+(1-y) f_{S_2}(x,y).
\label{avgfiteqs}
\end{align}
%
The time evolution of the ``Generous" types in both the species will give us the complete dynamics of the system.
However since the two interaction species are by definition different organisms, they can have different rates of evolution.
Thus if species 1 evolves at the rate $r_x$ while species 2 at rate $r_y$ then we have,
\begin{align}
\dot{x} &= r_x x \left(f_{G_1}(x,y) -  \bar{f}_1(x,y) \right)  \\
\dot{y} &= r_y y \left(f_{G_2}(x,y) -  \bar{f}_2(x,y) \right).
\label{eq:repeqapp}
\end{align}


\begin{sidewaysfigure}[ht]
    \includegraphics[width=\columnwidth]{../Figures/Dynamicsacrossp_reduced.pdf}
    \caption{
$d_1^{inter} = d_2^{inter} = 5$, $b = 2$, $r_x = r_y/8$, $M_1 = M_2 = 1$ and $c=1$ for the interspecies game. As for the intraspecies games we have $d_1^{intra} = d_2^{intra} = 5$ and $\tilde{b} = 10 $ with 
(a) $\tilde{c} = 3$, $\omega = 3/4$, 
(b) $\tilde{c} = 1$, $\omega = 3/4$, 
(c) $\tilde{c} = 1$, $\omega = 4/3$ and 
(d) $\tilde{c} = 3$, $\omega = 4/3$, recovering the traditional scenarios of dominance, coexistence and bistability \citep{hauert:JTB:2006a}.
\label{fig:appendix}
}
\end{sidewaysfigure}


\section{Population dynamics}

For brevity we begin with the description of population dynamics in species 1.
The two types in species 1, ``Generous" and ``Selfish" need not sum up to $1$ i.e. the population may not always be at its carrying capacity.
Hence if the empty space in the niche occupied by species $1$ is $z_1$, then we have $x_1 + x_2 + z_1 = $ where $x_1$ and $x_2$ are the densities of ``Generous" and ``Selfish" types.
The population dynamics then is dictated by,
%
\begin{align}
	\dot{x_1} &= r_x x_1 (z_1 f_{G_1} - e_1)  \\
	\dot{x_2} &= r_x x_2 (z_1 f_{S_1} - e_1) \\
	\dot{z_1} &= - \dot{x_1} - \dot{x_2} 
\end{align}
%
and for species 2
\begin{align}
	\dot{y_1} &= r_y y_1 (z_2 f_{G_2} - e_2)  \\
	\dot{y_2} &= r_y y_2 (z_2 f_{S_2} - e_2) \\
	\dot{z_2} &= - \dot{y_1} - \dot{y_2} 
\end{align}
%
We have $e_1$ and $e_2$ as the death rates for the two species. 
For the special case of  $e_1 = \frac{z_1 (x_1 f_{x_1} + x_2 f_{x_2}) }{x_1 + x_2}$ and $e_2 = \frac{z_2 (y_1 f_{G_2} + y_2 f_{S_2}) }{y_1 + y_2}$ we recover the two species replicator dynamics as in Eqs.~\ref{eq:repeqapp}. 
The fitnesses however need to be reevaluated in this setup.
For example in species 1 the fitness for type $G_1$ is,
%
\begin{align}
	f_{G_1}^{inter} &= \sum_{S=2}^{d_1} \binom{d_1 -1}{S-1} z_2 ^{d_1 -S} (1-z_2)^{S-1} P_G^{inter}(S,y_1,y_2,z_2) \\
	f_{G_1}^{intra} &= \sum_{S=2}^{d_1} \binom{d_1 -1}{S-1} z_1 ^{d_1 -S} (1-z_1)^{S-1} P_G^{intra}(S,x_1,x_2,z_1) \\
	f_{G_1} &= f_{G_1}^{inter} + f_{G_1}^{intra}
\end{align}
%
and similarly for type $S_1$ where the payoff functions are defined as,
%
\begin{align}
	P_G^{inter}(S,p,q,r) &= \sum_{k=0}^{S-1} V(S,p,q,r) \Pi_{G_1}(k+1) \\
	P_G^{intra}(S,p,q,r) &= \sum_{k=0}^{S-1} V(S,p,q,r) \Gamma_{G_1}(k+1) \\
	P_S^{inter}(S,p,q,r) &= \sum_{k=0}^{S-1} V(S,p,q,r) \Pi_{S_1}(k) \\
	P_S^{intra}(S,p,q,r) &= \sum_{k=0}^{S-1} V(S,p,q,r) \Gamma_{S_1}(k)
\end{align}
%
where $V(S,p,q,r) = \binom{S-1}{k} \left( \frac{p}{1-r}\right)^k  \left(\frac{q}{1-r}\right)^{S-1-k}$ is the probability of having a $k$ ``Generous"(Cooperator) individuals and $S-1-k$ ``Selfish"(Defector) individuals in the inter(intra) species game.
and the actual payoffs are calculated as per Eqs.~\ref{eqintergamepayoffs} and \ref{eqintragamepayoffs}.

\end{appendices}

\end{document}