\documentclass[times,onecolumn]{scrartcl} 
\usepackage{geometry}              
\usepackage{graphicx}
\usepackage{amssymb}
\usepackage{amsmath}
\usepackage{times}
\usepackage{color}
\renewcommand{\baselinestretch}{1.2}
\newcommand{\arne}[1]{\textcolor{red}{#1}}
\newcommand{\cha}[1]{\textcolor{blue}{#1}}
\newcommand{\leftsub}[2]{{\vphantom{#2}}_{#1}{#2}}



\begin{document}

\title{Supporting Text: \\
Population dynamics of mutualisms}

\author{
Chaitanya S. Gokhale 
}

\maketitle

\section{Average payoffs}
\subsection{Interspecies interactions}
\label{appA}

The two species are assumed to be in a mutualistic relationship.
Following \cite{bergstrom:PNAS:2003,souza:JTB:2009,gokhale:PRSB:2012}, we make use of the multiplayer version of the snowdrift game to represent a co-existence scenario.
Each species has two types of individuals ``Generous" ($G$) and ``Selfish" ($S$).
Say the frequency of $G$ individuals in species $1$ is given by $x$ and that in species 2 by $y$
Each individual from each species takes part in a $d$ player interspecies game where it interacts with $d-1$ individuals from the other species.
Of the $d-1$ the if $k$ of them are $G$ while $d-1-k$ are $S$ then the payoff accrued by the individuals is given by,
%
\begin{align}
f^{inter}_{G_1} (y) &= \sum_{k=0}^{d_1 -1} \binom{d_1 -1}{k}y^k (1-y)^{d_1 -1-k} \Pi_{G_1}(k+1) \\
f^{inter}_{S_1} (y) &= \sum_{k=0}^{d_1 -1} \binom{d_1 -1}{k}y^k (1-y)^{d_1 -1-k} \Pi_{S_1}(k).
\label{interfiteqs}
\end{align}
%
for the two types in species 1.
Analogously the fitness for the two types in species 2 can be written down as $f^{inter}_{G_2} (x)$ and $f^{inter}_{S_2} (x)$ which depend on the frequencies of $G$ in species 1.
The payoff for a $G$ thus includes itself ($k+1$) while for a $S$ individual only the the number of $G$ matter ($k$).
The payoffs themselves are defined as in \cite{souza:JTB:2009},
%
\begin{align}
\Pi_{S_1} (k) & = \begin{cases} b & \textrm{if } k \geq M \\ 0 & \textrm{if } k < M \end{cases}
\\
\Pi_{G_1} (k) & = \begin{cases} b-\frac{c}{k} & \textrm{if } k \geq M \\  -\frac{c}{M} & \textrm{if } k < M \end{cases}
\label{eqintergamepayoffs}
\end{align}
%
The selfish players get the benefit $b$ if the number of generous individuals in the interacting group, $k$, is greater than or equal to the threshold $M$.
For the generous individuals, their effort is subtracted from the payoffs.
The effort is shared if the quorum size is met ($\frac{c}{k}$), but is in vain for $k<M$.


\subsection{Intraspecies interactions}
\label{appB}

Within a species we do not assume a certain kind of interaction structure.
Instead we model the interaction matrix by making use of a general framework of costs and non-linear benefits \cite{eshel:AmNat:1988,hauert:JTB:2006a} which can potentially encompass all different types of social interaction structures qualitatively.
A crucial \textbf{assumption} which we make here is that the ``Generous" individuals from the between species interactions are the ``Cooperative" ones in the within species interaction and the ``Selfish" ones are the ``Defectors".
Hence for species 1 the frequency of cooperators is just $x$ and the defectors is $1-x$, the same as the ``Generous" and ``Selfish".
Again for simplicity we \textbf{assume} a $d$ player game being played within species, the same as in between species.
Thus the fitnesses of cooperators and defectors are defined as \cite{hauert:JTB:2006a},
%
\begin{align}
	f^{intra}_{G_1} (x) &= \sum_{k=0}^{d_1 -1} \binom{d_1 -1}{k}x^k (1-x)^{d_1 -1-k} \Gamma_{G_1}(k+1) \\
	f^{intra}_{S_1} (x) &= \sum_{k=0}^{d_1 -1} \binom{d_1 -1}{k}x^k (1-x)^{d_1 -1-k} \Gamma_{S_1}(k).
\label{intrafiteqs}
\end{align}
%
where the payoffs are given by,
\begin{align}
	\Gamma_{S_1} (k) = \frac{\tilde{b}}{d_1} \sum_{i=0}^{k-1} \omega^i \\
	\Gamma_{G_1} (k) = \Gamma_{S_1} (k) - \tilde{c}.
\label{eqintragamepayoffs}
\end{align}
%
Thus the defectors get a fraction of the benefit which is scaled by the factor $\omega$, which determines if the benefits are linearly accumulating ($\omega=1$) for increasing number of cooperators, synergistically enhanced ($\omega>1$) or saturating ($\omega<1$).
Note that the costs and benefits in the within species game need not be (and naturally so) the same as in between species ($b\neq \tilde{b}$ and $c \neq \tilde{c}$).


\subsection{Combined payoffs and dynamics}

The average payoffs are then just assumed to be a linear combination of the interspecies and intraspecies interactions where the parameter $p$ determines the strength of each of the interactions such that,
%
\begin{align}
	f_{G_1} (x,y) &= p f^{inter}_{G_1} (y) + (1-p) f^{intra}_{G_1} (x) \\
	f_{S_1} (x,y) &= p f^{inter}_{S_1} (y) + (1-p) f^{intra}_{S_1} (x)
\label{fiteqs}
\end{align}
%
Following the same procedure for the two strategies in species $2$ leads to the average fitness
%
\begin{align}
\bar{f}_1 (x,y) &= x f_{G_1} (y)+(1-x) f_ {S_1}(y)\\
\bar{f}_2 (x,y) &= y f_{G_2} (x)+(1-y) f_{S_2}(x).
\label{avgfiteqs}
\end{align}
%
The time evolution of the ``Generous" types in all the species will give us the complete dynamics of the system.
However since the two interaction species are by definition different organisms, they can have different rates of evolution.
Thus if species 1 evolves at the rate $r_x$ while species 2 at rate $r_y$ then we have,
\begin{align}
\dot{x} &= r_x x \left(f_{G_1}(y) -  \bar{f}_1(x,y) \right) \nonumber \\
\dot{y} &= r_y y \left(f_{G_2}(x) -  \bar{f}_2(x,y) \right).
\label{eq:repeqs}
\end{align}


\begin{figure*}[h]
\begin{center}
\includegraphics[width=\columnwidth]{../Figures/Dynamicsacrossp_reduced.pdf}
\caption{
$d_1 = d_2 = 5$, $b= 2$, $r_x = r_y/8$, $M_1 = M_2 = 1$ and $c=1$ for the interspecies game. As for the intraspecies games (a), (b), (c) and (d) the exact same parameter values as in \cite{hauert:JTB:2006a}.
}
\end{center}
\end{figure*}


\cha{\section{Asymmetries}}

\cha{This between and within species model is a powerful way of introducing a lot of variability into the dynamics,
\begin{align}
	d_1 &\neq d_2 \\
	d^{inter} &\neq d^{intra} \\
	M_1 &\neq M_2 \\
	b &\neq \tilde{b} \\
	c &\neq \tilde{c} \\
	r_x &\neq r_y \\
	&\vdots
\end{align}
and various combinations of these. We should justify why we don't do this here and why we do vary the ones that we do.}


\subsection{Dynamics in asymmetric conditions}

%We have addressed two kinds of asymmetries in the game, the number of player and the thresholds in the two species.
%We denote the number of players for species $1$ and species $2$ as $d_1$ and $d_2$, respectively, as in Fig.\ \ref{fig:counter}.
%That is if species $2$ is playing a $d_2$ player game it means that one player from species $2$ interacts with $d_2-1$ players of species $1$.
%For an asymmetry in the thresholds we use the two parameters $M_1\geq1$ and $M_2\geq1$ for the two species, respectively.

For asymmetric bimatrix games, there is a difference in the dynamics between the standard replicator dynamics and the 
alternative dynamics put forward by Maynard-Smith \cite{maynard-smith:1982to}.
For this dynamics, the average fitness of each species appears as a denominator,
\begin{align}
\dot{x} &= r_x x \left(f_{G_1}(y) -  \bar{f}_1(x,y) \right)/\bar{f}_1(x,y) \nonumber \\
\dot{y} &= r_y y \left(f_{G_2}(x) -  \bar{f}_2(x,y) \right)/\bar{f}_2(x,y).
\label{eq:repeqs}
\end{align}
In our asymmetric bimatrix game, the fixed point stability is affected by the choice of the dynamics, in contrast to the case of symmetric games. 
%In Fig.\ \ref{fig:thresholdsmodrep}, we illustrate that the dynamics is different between the usual replicator dynamics and Eqs. \ref{eq:repeqs}

For $d_1=d_2 \geq 5$, the exact coordinates of the fixed point must be computed numerically \cite{abel:AO:1824,stewart:book:2004}.


\section{Population dynamics}

For brevity we begin with the description of population dynamics in Species 1.
The two types in Species 1, ``Generous" and ``Selfish" need not sum up to $1$ i.e. the population may not always be at its carrying capacity.
Hence if the empty space in the niche occupied by Species $1$ is $z_1$, then we have $x_1 + x_2 + z_1 = $ where $x_1$ and $x_2$ are the densities of ``Generous" and ``Selfish" types.
The population dynamics then is dictated by,
%
\begin{align}
	\dot{x_1} &= r_x x_1 (z_1 f_{G_1} - e_1) \\
	\dot{x_2} &= r_x x_2 (z_1 f_{S_1} - e_1) \\
	\dot{z_1} &= - \dot{x_1} - \dot{x_2}
\end{align}
%
and for species 2
\begin{align}
	\dot{y_1} &= r_y y_1 (z_2 f_{G_2} - e_2) \\
	\dot{y_2} &= r_y y_2 (z_2 f_{S_2} - e_2) \\
	\dot{y_1} &= - \dot{y_1} - \dot{y_2}
\end{align}
%
We have $e_1$ and $e_2$ as the death rates for the two species.
Setting $e_1 = \frac{z_1 (x_1 f_{x_1} + x_2 f_{x_2}) }{x_1 + x_2}$ and $e_2 = \frac{z_2 (y_1 f_{G_2} + y_2 f_{S_2}) }{y_1 + y_2}$ we recover the two species replicator dynamics as in Eqs.~\ref{eq:repeqs}.
The fitnesses however need to be reevaluated in this setup.
For example in Species 1 the fitness for type $G_1$ is,
%
\begin{align}
	f_{G_1}^{inter} &= \sum_{S=2}^{d_1} \binom{d_1 -1}{S-1} z_2 ^{d_1 -S} (1-z_2)^{S-1} P_G^{inter}(S,y_1,y_2,z_2) \\
	f_{G_1}^{intra} &= \sum_{S=2}^{d_1} \binom{d_1 -1}{S-1} z_1 ^{d_1 -S} (1-z_1)^{S-1} P_G^{intra}(S,x_1,x_2,z_1) \\
	f_{G_1} &= f_{G_1}^{inter} + f_{G_1}^{intra}
\end{align}
%
and similarly for type $S_1$ where the payoff functions are defined as,
%
\begin{align}
	P_G^{inter}(S,p,q,r) &= \sum_{k=0}^{S-1} V(S,p,q,r) \Pi_{G_1}(k+1) \\
	P_G^{intra}(S,p,q,r) &= \sum_{k=0}^{S-1} V(S,p,q,r) \Gamma_{G_1}(k+1) \\
	P_S^{inter}(S,p,q,r) &= \sum_{k=0}^{S-1} V(S,p,q,r) \Pi_{S_1}(k) \\
	P_S^{intra}(S,p,q,r) &= \sum_{k=0}^{S-1} V(S,p,q,r) \Gamma_{S_1}(k)
\end{align}
%
where $V(S,p,q,r) = \binom{S-1}{k} \left( \frac{p}{1-r}\right)^k  \left(\frac{q}{1-r}\right)^{S-1-k}$ is the probability of having a $k$ ``Generous"(Cooperator) individuals and $S-1-k$ ``Selfish"(Defector) individuals in the inter(intra) species game.
and the actual payoffs are calculated as per Eqs.~\ref{eqintergamepayoffs} and \ref{eqintragamepayoffs}.

\bibliographystyle{pnas}
\bibliography{\string~/Bibtex/et.bib}

\end{document}