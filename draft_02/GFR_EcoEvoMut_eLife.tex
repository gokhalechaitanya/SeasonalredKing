\documentclass[12pt]{article}
\usepackage{times}
\usepackage[titletoc]{appendix}
\usepackage{graphicx}
\usepackage{lineno}
\usepackage{multirow}
\usepackage[english]{babel}
\usepackage{hyperref}
\hypersetup{
    colorlinks=true,       % false: boxed links; true: colored links
    linkcolor=blue,          % color of internal links (change box color with linkbordercolor)
    citecolor=darkgreen,        % color of links to bibliography
    filecolor=magenta,      % color of file links
    urlcolor= black           % color of external links
}
\usepackage{typearea} 
\usepackage{amssymb}
\usepackage{amsfonts}
\usepackage{amsmath}
\usepackage{enumerate}
\usepackage[round,authoryear]{natbib}

\renewcommand{\baselinestretch}{1.2}
\newcommand{\bbar}[1]{\overline{#1}}

\usepackage{color}
	 \definecolor{darkred}{rgb}{0.75,0,0}
	 \definecolor{darkgreen}{rgb}{0,0.5,0}
	 \definecolor{darkblue}{rgb}{0,0,0.75}
\newcommand{\cha}[1]{\textcolor{darkblue}{(#1)}}
\newcommand{\marcus}[1]{\textcolor{darkred}{(#1)}}
\newcommand{\paul}[1]{\textcolor{darkgreen}{(#1)}}

\title{\vspace*{-22mm}\bf Eco-evolutionary dynamics of mutualisms}
\author{Chaitanya S. Gokhale$^{1*}$,
Marcus Frean$^{2}$,
 \and Paul B. Rainey$^{1,3}$ \\
\normalsize $^{1}$New Zealand Institute for Advanced Study, Massey University, Auckland, New Zealand, \\
\normalsize $^2$Victoria University of Wellington, Wellington, New Zealand\\
\normalsize $^3$Max Planck Institute for Evolutionary Biology, \\
\normalsize August-Thienemann-Stra{\ss}e 2, 24306 Pl\"{o}n, Germany,\\
}

\date{}

\begin{document}

\linenumbers
\maketitle

\begin{abstract}
Mutualism has long been studied as a conundrum for evolutionary theory.
In the short run species that exploit other species would have a fitness advantage over a mutually costly relationship.
How such mutualisms are then maintained in the long run is a valid question.
However studies investigating this question often neglect the importance of within species interactions.
Separating inter and intraspecies interactions may not be always possible especially if the same individuals act within and between species.
Feedbacks between inter and intraspecies interactions are then inevitable and need to be taken into account.
Including population dynamics adds an ecological component to the study.
Herein we study the full eco-evolutionary dynamics of mutualism between two species when a variety of  intraspecies interactions are possible.
Our results show that while mutualism can turn into parasitism by overexploitation, for some intraspecies dynamics, mutualism can be maintained even while maintaining exploiters in the species composition.
\end{abstract}

\noindent
Keywords: mutualism,evolutionary game theory,multiple players, population dynamics, seasonality

\tableofcontents

\section{Introduction}

"Is there anything left to say about mutualisms .... The authors of this volume apparently think there is something to say, but I wonder if we are not beating a dead horse." 
- \cite{janzen:bookchapter:1985}.

Mutualisms have been debated over for a long time.
As with many concepts, we can trace back the study of mutualism to Aristotle \citep{aristotle:book:350}.
Formally the Belgian zoologist Pierre van Beneden coined the term mutualism in $1873$ \citep{bronstein:book:2003}.
The study of mutualistic relationships, interspecific interactions that benefit both species, is rich in empirical as well as theoretical understanding 
\citep{boucher:book:1985,hinton:PTENHS:1951,wilson:AmNat:1983,bronstein:QRB:1994,pierce:ARE:2002,kiers:Nature:2003,bshary:ASB:2004} \citep{poulin:JTB:1995,doebeli:PNAS:1998,noe:book:2001,johnstone:ECL:2002,bergstrom:PNAS:2003,hoeksema:AmNat:2003,akcay:PRSB:2007,bshary:Nature:2008}.
Most examples of mutualisms lend themselves to the idea of direct reciprocity \citep{trivers:QRB:1971} and have thus been extensively studied using evolutionary game theory.
Classical evolutionary games are usually limited to dyadic interactions \citep{weibull:book:1995,hofbauer:JMB:1996,hofbauer:book:1998}.
Between two species then, the fundamental interaction is between two individuals, one from each species, and the sum of many such interactions determines the evolutionary dynamics.
However, this is clearly a simplification as has been shown by numerous studies \citep{noe:TREE:1995,noe:book:2001,kiers:Nature:2003,stanton:AmNat:2003,stadler:book:2008}.

A well studied example of a one-to-many interaction is that of the plant-microbe mutualism wherein leguminous hosts prefer rhizobial symbionts that fix more nitrogen \citep{kiers:Nature:2003}, or where plants provide more carbon resources to fungal strains that are providing better access to nutrients \citep{kiers:Science:2011}.
As an example of an animal host, mutualistic relationship between the bioluminescent bacteria \textit{Vibrio fischeri} and \textit{Euprymna scolopes}, the bobtail squid \citep{mcfallngai:PLoSB:2014} is a paradigm. 
Numerous bacteria are hosted in the crypts of the squid's light organ, where they produce light despite it being costly to do so. 
The bacteria mature and develop within the squid, however those that fail to produce bioluminescence are evicted. 
While the variation in the phenotypes of the interacting partners has been acknowledged, the usual analysis focuses on the interaction between the two species without addressing this additional complexity.
The classic example of ants and aphids or butterfly larvae \citep{pierce:BES:1987,hoelldobler:book:1990} is an excellent exposition of many player interactions. Numerous ants tend to each of the soft bodied creatures, providing them with shelter and protection from predation and parasites, in exchange for honeydew, a rich source of food for the ants \citep{hill:OEC:1989,stadler:book:2008}.
This is a one-to-many interaction from the perspective of the larva.

While inferring the particular type of interspecific symbiosis (mutualism, parasitism or commensalism) might be possible, identifying and quantifying the underlying intraspecific variation can be a daunting task \citep{behm:JE:2014}.  
Intraspecific interactions are usually studied in isolation and separate from the interspecies relationships.
For example while cohorts of cleaner fish together have been taken to determine the quality of a cleaning station \citep{bshary:AB:2002,bshary:book:2003}, this can also drive variation of quality of cleaning within a cleaning station via interactions of individual cleaner fish amongst themselves.  
In this manuscript we look at the broader picture of how the evolutionary
dynamics within a species are shaped when both the inter as well as intraspecies
dynamics are taken together. 
We find that including the full range of interactions provides us with a set of rich and intricate dynamics which are not possible when either one of these dimensions is ignored.

Mutualistic relationships are, by definition, between species, and timing may be crucial for their maintenance. 
It is natural to imagine that the observed mutualism may be seasonal and the interactions are not a continuous feature of the evolutionary trajectory of a species. 
Three species of spiderhunter sunbirds, \textit{Arachnothera}, pollinate the evergreen subtropical ``lipstick plant", \textit{Aeschynanthus speciosus}, only twice a year.
A wildly changing ecology can affect the flowering time of some plants and the maturation of the dispersers they depend on, easily disrupting such delicately balanced mutualistic interactions. 
Unless both interacting species can respond in a similar fashion such a mutualism will break down \citep{warren:GCB:2014}.
We tackle this seasonality by varying the duration of the impact of intraspecies and interspecies dynamics.

To complete the ecological picture we include population dynamics to the evolutionary process of the mutualists.
The two species can often occupy different niches and 
Such dynamics informs us about the population densities we might expect to find the interactors to evolve to.
We demonstrate the crucial nature of the feedback between population and evolutionary dynamics which can maintain mutualisms preventing either or both species from going extinct. 
Beginning with the previously studied interspecies dynamics as the foundational framework \citep{gokhale:PRSB:2012} we increase the complexity of the system by including intraspecies dynamics, and then seasonality.
For the complete eco-evolutionary picture to emergy we include population dynamics next.
The rich dynamics observed provides us with novel insights about the immense asymmetries in mutualisms and the fragility of such delicately balanced interactions.



\section{Model and Results}


\begin{figure}
\begin{center}
\includegraphics[scale=0.5]{Figures/interintra.pdf}
\caption{\small{
\textbf{Evolutionary dynamics with combined inter-intra-species dynamics.}
We assume the interactions between species to be mutualistic described by the snowdrift game \citep{bergstrom:PNAS:2003,souza:JTB:2009,gokhale:PRSB:2012}.
Species $1$ plays a $d_1^{inter}$ player game with species $2$ while species $2$ plays a $d_2^{inter}$ player game.
Each species has two types of players ``Generous" and ``Selfish" who besides interacting with the members of other species, also take part in intraspecies dynamics.
For intraspecies interactions we assume a general framework of synergy and discounting which can recover the \textit{classical} outcomes of evolutionary dynamics\citep{eshel:AmNat:1988,hauert:JTB:2006a,nowak:book:2006}
}
\label{fig:conceptart}
}
\end{center}
\end{figure}


\subsection{Interaction dynamics}
\subsubsection{Interspecies}

Focusing on mutualism, the interspecies dynamics is given by the multiplayer version of the snowdrift game \citep{bergstrom:PNAS:2003,souza:JTB:2009,gokhale:PRSB:2012} (also known as hawk-dove, or chicken).
A common benefit is generated by contributions from both species but there is a cost involved to it and species do not need to contribute equally. 
However the individuals in each species could get away with contributing a bit less than other individuals.
Hence for example if producing brighter light comes at a premium for the \textit{Vibrio} in the squid then the dimmer \textit{Vibrio} would be better off (Not producing any light is not an option as the squid then actively evicts these bacteria) \citep{mcfallngai:PLoSB:2014}.
We assume that each species consists of two types of individuals ``Generous" $G$ and ``Selfish" $S$. 
If enough individuals are ``Generous" and contributing to the generation of mutual benefits then other individuals can get away with being selfish (not contributing). 
But all individuals in the game lose out if not enough are generous. Hence both species cannot be completely ``selfish", as per the definition of mutualism.
This interaction framework corresponds to that of a multiplayer version of a snowdrift game and is discussed in detail in the Supplementary Material (SI).
Hence the pressure is on a species to make the partner ``Generous" while itself being ``Selfish".
The fitness of each of the types within a species depends on the composition of the other species.
Denoting the frequency of the``Generous" types in species $1$ ($G_1$) as $x$, and that in species $2$ ($G_2$) as $y$, the fitness of $G_1$  is given by $f^{inter}_{G_1} (y)$ and that of $G_2$ as $f^{inter}_{G_2} (x)$. 

\subsubsection{Intraspecies}

For intraspecies dynamics we do not restrict ourselves to any particular interaction structure and thus make use of the general multiplayer evolutionary games framework \citep{gokhale:PNAS:2010,gokhale:DGAA:2014}.
Moving from the interspecies dynamics, the two types already described are ``Generous" and ``Selfish".
Thus we already have each species containing two different types of individuals.
It is possible that a different categorisation exists within a species.
Thus if the interactions within a species are say between ``Cooperators" and ``Defectors", these types could be made up of a combination of ``Generous" and ``Selfish" individuals.
However for the sake of simplicity we study the dynamics between ``Generous" and ``Selfish" types within a species where the types are defined at the interspecies level.
The cost benefit framework described in \citep{eshel:AmNat:1988,hauert:JTB:2006a}
 allows us to transition between four classic scenarios of evolutionary dynamics \citep{nowak:Science:2004}.
For example in our case we can have a dominance of the ``Generous" type or the ``Selfish" type or both the types can invade from rare resulting in a co-existence or bistability if both pure strategies are mutually non-invasive.
For the intraspecies interactions the fitness of a $G_1$ is then given by $f^{intra}_{G_1} (x)$ and that of $G_2$ is given by $f^{intra}_{G_2} (y)$ and similarly for the ``Selfish" types.

\subsection{Combined dynamics}

Putting together intra and interspecific dynamics provides a complete picture of the possible interactions occurring. While we are interested in mutualism at the level of the interspecies interactions there are four possible interactions within each species \citep{nowak:Science:2004,hauert:JTB:2006a} (dominance of either type, coexistence or bistability). Since the within species interactions for the two different species do not need to be the same, there are in all sixteen different possible combinations.
Assuming additivity in the fitnesses of inter and intraspecies fitnesses, the combined fitness of each of the two types in the two species are given by,

%
\begin{align}
\label{eq:combinedfiteqs}
	f_{G_1} (x,y) &= p f^{inter}_{G_1} (y) + (1-p) f^{intra}_{G_1} (x) \nonumber \\
	f_{S_1} (x,y) &= p f^{inter}_{S_1} (y) + (1-p) f^{intra}_{S_1} (x) \nonumber \\
	f_{G_2} (x,y) &= p f^{inter}_{G_2} (x) + (1-p) f^{intra}_{G_2} (y) \\
	f_{S_2} (x,y) &= p f^{inter}_{S_2} (x) + (1-p) f^{intra}_{S_2} (y) \nonumber
\end{align}
%
The parameter $p$ tunes the impact of each of the interactions on the final fitness that eventually drives the evolutionary dynamics.
For $p=1$ we recover the well studied case of the Red King dynamics \citep{gokhale:PRSB:2012}, while for $p=0$ the dynamics of the two species are  decoupled and can be individually studied by the synergy/discounting framework of nonlinear social dilemmas \citep{hauert:JTB:2006a}.
Of interest here is the continuum described by the intermediate values of $p$.
However that means we need to track the qualitative dynamics of sixteen possible intraspecies dynamics as $p$ changes gradually from $0$ to $1$ (Appendix \ref{app:combineddyn}). 
The time evolution of the ``Generous" types in both species is then given by,
%
\begin{align}
\dot{x} &= r_x x \left(f_{G_1}(x,y) -  \bar{f}_1(x,y) \right) \nonumber \\
\dot{y} &= r_y y \left(f_{G_2}(x,y) -  \bar{f}_2(x,y) \right).
\label{eq:repeq}
\end{align}
%
This approach provides us with a powerful method to incorporate a multitude of realistic concepts in the analysis.
For example the number of players involved in a game, which has been shown to be a crucial factor in determining the evolutionary dynamics could be different for each interactions, inter and intraspecies interactions for species $1$ ($d^{inter}_1$, $d^{intra}_1$) and similarly for species 2 ($d^{inter}_2$, $d^{intra}_2$). 
The interspecies interactions are proxied by the multiplayer snowdrift game which can incorporate threshold effects.
For example a certain number of ``Generous" cleaner fish may be required to clean the host or a certain number of ``Generous" ants required to protect larva from predators.
We can have $M_1$ and $M_2$ as the thresholds in the two species.
Since the interaction matrices for the inter and intraspecies dynamics are completely different, in principle, we can have different costs and benefits for the four games (two snowdrift games from the perspective of each species and the intragames within each species).

We can have a diverse and rich set of dynamics possible which brings into question the study of coevolution based on only interspecies interactions. 
For a given set of parameter values but the the full spectrum of possible dynamics, see Figure \ref{fig:appendix}.
Even under a large number of assumptions and even if the intraspecies dynamics accounts for only $33\%$ ($1-p$) of the cumulative fitness, we can see drastically different qualitative dynamics which is capable of explaining the persistence of exploiters (Fig. \ref{fig:mainexampleone}).

\begin{figure}
\begin{center}
\includegraphics[width=\columnwidth]{Figures/mainexample2.pdf}
\caption{\small{
\textbf{Change in evolutionary dynamics due to inclusion of intraspecies dynamics.} When the fitness of the ``Generous" and ``Selfish" types in both the species is solely determined by the interactions which occur between species (in this case mutualism, $p=1$) then we recover the dynamics as studied previously in \citep{gokhale:PRSB:2012}. The colours represent the initial states which result in an outcome favourable for species $1$ (blue leading to ($S_1$,$G_2$)) and species $2$ (red, leading to ($G_1$,$S_2$)). This can result in the Red King effect and other possible complexities as discussed recently in \citep{gao:SciRep:2015}. However when we start including intraspecies dynamics the picture can be very different.
Even when the impact of intraspecies dynamics is only a $1/3$ on the total fitness of the ``Generous" and ``Selfish" types we see a very qualitatively different picture.
Two fixed points are observed where both the ``Generous" and ``Selfish" types can co-exist in both the species.
All initial states in the interior lead to either one of these fixed points (hence the lack of colours).
However it is still possible to characterise the ``successful" species as one of the equilibrium is favoured by one species than the other.
The horizontal isoclines are for species $1$ while the vertical ones are for species $2$.
The analysis was done for a 5 player game $d_1^{inter} = d_2^{inter} = d_1^{intra} = d_2^{intra} = 5$, $b=2$, $c=1$ and $r_x = r_y /8$ for the interspecies mutualism game while additionally $\tilde{b}_1 = \tilde{b}_2 = 10$ and $\tilde{c}_1 = \tilde{c}_2 = 1$ and $\omega_1 = \omega_2 = 3/4$ for the two intraspecies games within each species. Note that even with symmetric games within each species we can a qualitatively drastic difference when compared to the dynamics excluding intraspecies interactions.  For different intraspecies interactions within each species and for varying $p$ see SI.}
\label{fig:mainexampleone}
}
\end{center}
\end{figure}

\subsection{Seasonality}

Many mutualisms are observed only during certain periods of a year.
Such seasonal or episodic mutualism run a high risk of phenological partner mismatch as a result of climate change \citep{rafferty:Oikos:2015}.
While tropical species, such as the various varieties of fig (\textit{Ficus}) can flower all year round, their mutualistic relationships (with wasps) run a lower risk.
For example in the ant-aphid mutualism, the number of attending ants was seen to increase till June and declined after late July and the aphid colonies went rapidly (within a month) extinct in the absence of attending ants \citep{yao:Oikos:2000,yao:JIS:2009}.
For the evolution of a species this means that the effect of interspecific interaction changes over time.

To analyse such episodic mutualistic events, instead of a static variable $p$ measuring the impact of interspecific interaction on fitness we make use of a time-dependent function $p(t) = (1 + \sin(a t))/2 $.
For the particular parameter set used in Fig. \ref{fig:mainexampleone} ($p=0.666$ panel), introducing seasonality still maintains the two interior fixed points (they are closer to each other for $p = 0.5$), but this is seen only when the oscillations in $p(t)$  are comparable, $a=1$, or faster, $a=10$, with respect to the evolutionary timescale.
For slower oscillations $a=0.1$ we see cyclic behaviour which is prominent in species $2$ more than in species $1$.
Very slow oscillations mean that the system spends longer close to the starting value of $p(t)$ and hence the phase in which $p(t)$ starts becomes more and more important for smaller and smaller $a$. 
This is especially interesting if the stability of the system is qualitatively affected over the $p$ continuum.


\begin{figure}
\begin{center}
\includegraphics[width=\columnwidth]{Figures/oscillating_p_reduced.pdf}
\caption{\small{
\textbf{Seasonal changes in the interspecies interactions affecting the evolutionary dynamics within species.}
We model the impact of the interspecies interaction on the fitness of the different types as in Eqs.\ref{eq:combinedfiteqs} however instead of a static value for $p$ we introduce seasonality via a simple sine function as $p(t) = (1+\sin(at))/2$.
Here, $a$ denotes how the seasonality time scale relates to the inter-intra-species interactions timescale.
A large $a$ denotes multiple bouts of mutualism affecting fitness for a given evolutionary time step while a small $a$ denotes fewer of such bouts within the same evolutionary time step.
The trajectories shown in the panels are obtained by numerical interactions with initial conditions $x = y = \{0.1,0.9\}$ and a step size of $\Delta x = \Delta y = 0.1$.
The background colour is obtained by a finer grain of $\Delta x = \Delta y = 0.01$ and depict the same outcomes as in Fig. \ref{fig:mainexampleone}, with gray representing the outcome that none of the edge equilibria are reached.
For comparable or larger $a$ the dynamics under oscillations can be captured by the average dynamics (at $p = 0.5$) however for small $a$ we a see qualitatively different outcome.
Furthermore the phase in which the oscillating function begins is more important for smaller and smaller $a$ especially if the stability of the fixed points changes as $p$ changes (see Fig. \ref{fig:appendix} panel (b) x (b) across the $p$ continuum).
 }
\label{fig:oscillations}
}
\end{center}
\end{figure}


\subsection{Population dynamics}

Until now we have considered that each species consists of two types of individuals and they make up the population of that species.
However populations sizes change over time. 
Assuming that ecological changes are fast enough that they can be averaged out, we can usually ignore their effect on the evolutionary dynamics.
It is now possible to show that evolution can happen at fast timescales, comparable to those of the ecological dynamics \citep{post:PTRSB:2009,beaumont:Nature:2009,hanski:PNAS:2011,sanchez:PLoSB:2013}.
Hence we need to tackle not just evolutionary but eco-evolutionary dynamics together.
%
\begin{figure}
\begin{center}
\includegraphics[scale=0.5]{Figures/popdyninterintra.pdf}
\caption{\small{
\textbf{Population and evolutionary dynamics with combined inter-intra-species dynamics.}
As with the interactions described in Fig. \ref{fig:conceptart} the two species consist of two types of individuals ``Generous" and ``Selfish".
Since the two species can in principle occupy different environmental niches, they  can have non-overlapping population carrying capacities.
The normalised carrying capacity in both species is $1$ and we have $x_1 + x_2 + z_1 = 1$ (for species $1$) where $x_1$ and $x_2$ are the densities of the ``Generous" and ``Selfish" types respectively (similarly with $y$ and $z_2$ in species $2$). 
The parameter $z_1$ represents the remaining space into which the population can still expand into.
For $z_1 = 0$ the species $1$ is at its carrying capacity while for $z_1 = 0$ it is extinct.}
\label{fig:conceptartpopdyn}
}
\end{center}
\end{figure}
%

To include population dynamics in the previously considered scenario, we reinterpret $x_1$ now as the fraction of ``Generous" types and $x_2
$ as the fraction of ``Selfish" types in species $1$.
Further we have $z_1 = 1 - x_1 - x_2$ as the empty spaces in the niche occupied by species $1$. 
Similarly we have $y_1$, $y_2$ and $z_2$ (Fig.~\ref{fig:conceptartpopdyn}).
This approach has previously been explored in terms of social dilemmas in \citep{hauert:PRSB:2006}.
We adapt and modify it for two species and hence now the dynamics of this complete system is determined by the following set of differential equations,
%
\begin{align}
	\dot{x_1} &= r_x x_1 (z_1 f_{G_1} - e_1) \nonumber \\
	\dot{x_2} &= r_x x_2 (z_1 f_{S_1} - e_1) \\
	\dot{z_1} &= - \dot{x_1} - \dot{x_2} \nonumber
\end{align}
%
for species 1, and
%
\begin{align}
	\dot{y_1} &= r_y y_1 (z_2 f_{G_2} - e_2) \nonumber \\
	\dot{y_2} &= r_y y_2 (z_2 f_{S_2} - e_2) \\
	\dot{z_2} &= - \dot{y_1} - \dot{y_2} \nonumber
\end{align}
%
for species 2. 
We have introduced $e_1$ and $e_2$ as the death rates of the two species.
Setting $e_1 = \frac{z_1 (x_1 f_{x_1} + x_2 f_{x_2}) }{x_1 + x_2}$ and $e_2 = \frac{z_2 (y_1 f_{G_2} + y_2 f_{S_2}) }{y_1 + y_2}$ we recover the two species replicator dynamics as in Eqs.~\ref{eq:repeq} (For the sake of brevity we avoid showing the fitnesses in their the functional forms).
In this setup however the fitnesses need to be re-evaluated as now we need to account for the presence of empty spaces (See SI).
The dynamics is simplified by focusing on the proportion of ``Generous" types in both the species thus $g_1 = x_1/(1-z_1)$ and $g_2 = y_1/(1-z_2)$ whose time evolution is given by,
\begin{align}
	\dot{g_1} &= r_x z_1 g_1 (1-g_1) (f_{G_1} - f_{S_1}) \nonumber \\
	\dot{z_1} &= e_1 (1-z_1) - r_x z_1 (1-z_1) (g_1 f_{G_1} -  (1-g_1) f_{S_1})
\end{align}
and
\begin{align}
	\dot{g_2} &= r_y z_2 g_2 (1-g_2) (f_{G_2} - f_{S_2}) \nonumber \\
	\dot{z_2} &= e_2 (1-z_2) - r_y z_2 (1-z_2) (g_2 f_{G_2} -  (1-g_2) f_{S_2})
\end{align}
%
where everywhere we have $x_1 = g_1 (1-z_1)$ (with $x_2 = (1-g_1) (1-z_1)$) and $y_1 = g_2 (1-z_2)$ (with $y_2 = (1-g_2) (1-z_2)$) in the fitnesses as well.

Interactions at varying population densities affect the group size formation which now includes the possibilities of player positions being left empty.
Thus for smaller population densities the interactions groups are small and vice versa for lager densities.
Effect of group size on the evolutionary dynamics is a well documented phenomena which can potentially change the results qualitatively \citep{pacheco:PRSB:2009,souza:JTB:2009}.
Such a two species multi-type interaction system is a complicated as well as a realistic depiction of most of the mutualisms observed in nature.
However given this complexity, we need to look at the dynamics within the two species simultaneously.

We take the most stable situation observed in the dynamics when population dynamics is absent (Fig.~\ref{fig:mainexampleone}) which shows two internal stable equilibria and add population dynamics to it.
The results are summarised in Figure \ref{fig:popdyn} where we plot the evolutionary parameter (fraction of ``Generous" in each species) against the ecological parameter, the population density (or rather in this case the empty spaces) .

\begin{figure}
\begin{center}
\includegraphics[width=\columnwidth]{Figures/mainexamplepopdyn2.pdf}
\caption{\small{
\textbf{Dynamics of evolutionary strategies and population density for an intraspecies coexistence game with interspecies mutualism.}
With exactly the same parameters as that of Figure \ref{fig:mainexampleone} with  symmetric death rates $e_1 = e_2 = 0.05$ we show two different numerically evaluated examples.
Left Panel: shows the outcomes in species $1$ when starting from $0.5$ fraction of ``Generous" individuals in species $2$ at half carrying capacity $z_2 = 0.5$.
While most of the initial conditions lead to an extinction of species $1$ (red), there exists a fixed point which can be reached when most of species $1$ is ``Generous" and close to carrying capacity (green). For the same or higher fraction of $G_1$ but lower population density, species $1$ can end up being completely ``Generous" (blue).
Right Panel: shows the outcomes in species $2$ when starting from $0.6$ fraction of ``Generous" individuals in species $1$ with empty spaces proportion of $z_1 = 0.6$.
When species $2$ is mostly made up of ``Selfish" types then it leads to species extinction (red), For intermediate levels of ``Generous"individuals there exists an internal equilibrium (dark green). However another stable equilibrium exists as well as even higher densities of ``Generous" types closer to full carrying capacity (green).
Equilibrium selection is thus possible for species $2$ in this case where it is preferable to have an intermediate number of ``Selfish" types.
\label{fig:popdyn}
}
}
\end{center}
\end{figure}



\section{Discussion}

Mutualistic interactions have been implicated as one possible mechanism facilitating the success of invasive species \citep{richardson:BR:2000}.
However mutualism based invasions also have the possibility to change the composition of the supporting local species.
About 150 ant species have been found invading new habitats mostly by forming mutualistic associations with honeydew producers \citep{mcglynn:JB:1999}.

Usually when interspecies relationships such as mutualism (or antagonist relationships as in predator-prey) are considered, the within species interactions are ignored for the sake of convenience.
The converse is the case when the intraspecies interactions are of interest.
The major body of work focusing on within population social dilemmas between Cooperators" and Defectors" is an example of the same.
Obviously, this is an assumption which is very useful when distilling the interactions at different community scales.
However when the inter and intraspecies interactions are interdependent then the feedbacks between the two levels cannot be ignored \citep{schluter:PlosB:2012}.

In principle the framework developed herein is capable of handling a diverse array of inter and intraspecific interactions.
For interspecific interactions our focus is on mutualism.
Mutualistic interactions between two species can be represented by a bimatrix game.
The components of each of the two game matrices need not be correlated as long as they independently satisfy the inequalities leading to a Snowdrift games. Including realistic phenomena such as intraspecies interactions, population dynamics and seasonality we show that maintenance of mutualism is possible.
A fragile balance of parameters maintains mutualism. If within each species the Generous" and Selfish" interactions result in coexistence then it can outweigh the competition which they experience at the interspecies level.
Note however that at the interspecies level the competition of a Generous" individuals is with the Selfish" individuals from the other species. While the Selfish" individuals from the other species can drive Generous" individuals within a species extinct, co-existence between Generous" and Selfish" within the same species can overcome the pressure for extinction. In this way mutualism can be maintained but it comes at a cost of also maintaining a significant level of exploiters. In fact the coevolutionary dynamics between the two species is determined together by the inter as well as the intraspecific interactions.

While the simple case makes predictions possible, including seasonality inserts a time dependent factor which makes analytical reasoning difficult.
However given the patterns of episodic interactions and studies of mutualistic relationship obtained from field studies over the decades it might be possible to include the seasonal component in future analysis on realistic systems to see how the interactions are going to change under drastic climate change.
Including this feature informs us of the dependence of the mutualism on environmental factors.
This is particularly important as species specific mutualisms are at a high risk of being destabilised.
For example bird pollinators of numerous plants are sensitive to the environmental and ecological changes which can occur naturally or catalysed by anthropogenic activity.
The difference in the timescales of the evolutionary process and environmental fluctuations highlights the fact that averaging out the environmental effects might not always be possible.
The system can show qualitatively different behaviour from the average dynamics depending on the kind of interactions initially involved within and between species.


An ecologically important example of species specific mutualism is that of sunbirds, particularly the Malachite Sunbird (\textit{Nectarine famosa}) and the geophyte \textit{Brunsvigia littoralis}.
Besides being sensitive to the environmental variation, \textit{B. littoralis} it furthermore suffers from low population densities \citep{geerts:SAJB:2012} and threatened by rapid urbanisation.
An example more economically connected to humans comes from the honeybee, \textit{Apis mellifera}.
A species of immense capital importance, the colony collapse of this pollinator has been attributed to numerous causes ranging from the pesticides to biological interference from parasites and pathogens as well as a change in the environment \citep{nazzi:PLosPath:2012}.
Our framework incorporates exactly these essential elements of interspecies interactions and changing environments, predicting a deep impact on population dynamics of the interactors.
While our focus is currently on mutualism, it is easy to change the interactions by changing/modifying the game defining the interspecies interactions.
Including population dynamics and the real threat of extinction needs to be acknowledged when modelling such scenarios.
Only then can better conservation tactics be formulated which are not solely based on evolutionary predictions but eco-evolutionary dynamics.

Going back to one of the most well studied examples of mutualism, the squid-vibrio symbiosis, it is hard to exclude population dynamics at least from one of the interacting species \citep{nyholm:NRM:2004}.
The diel pattern of the host squid is associated with oscillations in the population density of the symbiont \textit{Vibrio fiscerei}.
Since the growth rates of the two species differ vastly, the population size turnover inside the squid needs to be managed.
While the squid makes use of the full light organ at night camouflaging itself from predators, at dawn it expels almost $95\%$ of the bacteria.
The squid lies buried underground during the day and the remaining $5\%$ of the bacteria repopulate the light organ again reaching saturation by mid-afternoon.
While in our model the population dynamics of the mutualists are driven by the empty spaces in their own niches, this particular example beckons a specific modeling approach where the host itself acts as the niche environment of the symbiont and controls the population density too.
Our framework comes with the capability of including such specific examples and can be modified to suit particular examples.
It thus helps in not only elucidating the interactions which might be involved in generating the dynamics which we observe in nature but rather provide the criteria under which the observed dynamics are being maintained and ways to explore their stability under varying crucial parameters. 

Our study shows the critical nature of mutualism and the sliver of parameter space where they are maintained. 
A slight change in the values can either end up in a system where one of the mutualist is completely exploited by the other species or even leads to extinction of both types in case of obligate mutualisms.
Going back to \cite{janzen:bookchapter:1985} an appropriately succinct summary would be,
`A mutualist today may be a parasite of the mutualism tomorrow'.

\textbf{Acknowledgements}. \cha{CSG acknowledges funding from the New Zealand Institute for Advanced Study and time spent at Victoria University of Wellington. \ldots }


\bibliographystyle{plainnat}
%\bibliographystyle{mdpi}

%\bibliography{\string~/Bibtex/et.bib}

\begin{thebibliography}{61}
\providecommand{\natexlab}[1]{#1}
\providecommand{\url}[1]{\texttt{#1}}
\expandafter\ifx\csname urlstyle\endcsname\relax
  \providecommand{\doi}[1]{doi: #1}\else
  \providecommand{\doi}{doi: \begingroup \urlstyle{rm}\Url}\fi

\bibitem[Ak{\c c}ay and Roughgarden(2007)]{akcay:PRSB:2007}
E.~Ak{\c c}ay and J.~Roughgarden.
\newblock Negotiation of mutualism: rhizobia and legumes.
\newblock \emph{Proceedings of the Royal Society B}, 274:\penalty0 25--32,
  2007.

\bibitem[{Aristotle (Translator - Allan Gotthelf)}(1991)]{aristotle:book:350}
{Aristotle (Translator - Allan Gotthelf)}.
\newblock \emph{History of Animals}.
\newblock Number {Books VII-X. No - 439} in Loeb Classical Library. Harvard
  University Press, 1991.

\bibitem[Beaumont et~al.(2009)Beaumont, Gallie, Kost, Ferguson, and
  Rainey]{beaumont:Nature:2009}
H.~J.~E. Beaumont, J.~Gallie, C.~Kost, G.~C. Ferguson, and P.~B. Rainey.
\newblock Experimental evolution of bet hedging.
\newblock \emph{Nature}, 462:\penalty0 90--93, 2009.

\bibitem[Behm and Kiers(2014)]{behm:JE:2014}
Jocelyn~E Behm and E~Toby Kiers.
\newblock {A phenotypic plasticity framework for assessing intraspecific
  variation in arbuscular mycorrhizal fungal traits}.
\newblock \emph{Journal of Ecology}, 102\penalty0 (2):\penalty0 315--327, 2014.

\bibitem[Bergstrom and Lachmann(2003)]{bergstrom:PNAS:2003}
C.~T. Bergstrom and M.~Lachmann.
\newblock The {R}ed {K}ing {E}ffect: When the slowest runner wins the
  coevolutionary race.
\newblock \emph{Proceedings of the National Academy of Sciences USA},
  100:\penalty0 593--598, 2003.

\bibitem[Boucher(1985)]{boucher:book:1985}
D.~H. Boucher.
\newblock The idea of mutualism, past and future.
\newblock In D.~H. Boucher, editor, \emph{The Biology of Mutualism}, pages
  1--28. Oxford University Press, New York, 1985.

\bibitem[Bronstein(1994)]{bronstein:QRB:1994}
J.~L. Bronstein.
\newblock Our current understanding of mutualism.
\newblock \emph{The Quarterly Review of Biology}, 69:\penalty0 31--51, 1994.

\bibitem[Bronstein(2003)]{bronstein:book:2003}
J.~L. Bronstein.
\newblock Exploitation within mutualistic interactions.
\newblock In Peter Hammerstein, editor, \emph{Genetic and Cultural Evolution of
  Cooperation}. MIT Press, 2003.

\bibitem[Bshary and Sch{\"a}ffer(2002)]{bshary:AB:2002}
R.~Bshary and D.~Sch{\"a}ffer.
\newblock Choosy reef fish select cleaner fish that provide high-quality
  service.
\newblock \emph{Animal Behaviour}, 63:\penalty0 557--564, 2002.

\bibitem[Bshary et~al.(2008)Bshary, Grutter, Willener, and
  Leimar]{bshary:Nature:2008}
R.~Bshary, A.~S. Grutter, A.~S.~T. Willener, and O.~Leimar.
\newblock Pairs of cooperating cleaner fish provide better service quality than
  singletons.
\newblock \emph{Nature}, 455:\penalty0 964--967, 2008.

\bibitem[Bshary and Bronstein(2004)]{bshary:ASB:2004}
R.~S. Bshary and J.~L. Bronstein.
\newblock Game structures in mutualisms: what can the evidence tell us about
  the kinds of models we need?
\newblock \emph{Advances in the Study of Behavior}, 34:\penalty0 59--104, 2004.

\bibitem[Bshary and No{\"{e}}(2003)]{bshary:book:2003}
R.~S. Bshary and R.~No{\"{e}}.
\newblock Biological markets: the ubiquitous influence of partner choice on the
  dynamics of cleaner fish-client reef fish interactions.
\newblock In Peter Hammerstein, editor, \emph{Genetic and Cultural Evolution of
  Cooperation}, pages 167--184. MIT Press, 2003.

\bibitem[Doebeli and Knowlton(1998)]{doebeli:PNAS:1998}
M.~Doebeli and N.~Knowlton.
\newblock The evolution of interspecific mutualisms.
\newblock \emph{Proceedings of the National Academy of Sciences USA},
  95:\penalty0 8676--8680, 1998.

\bibitem[Eshel and Motro(1988)]{eshel:AmNat:1988}
I~Eshel and U~Motro.
\newblock {The three brothers' problem: kin selection with more than one
  potential helper. 1. The case of immediate help}.
\newblock \emph{American Naturalist}, pages 550--566, 1988.

\bibitem[Gao et~al.(2015)Gao, Li, and Wang]{gao:SciRep:2015}
Lei Gao, Yao-Tang Li, and Rui-Wu Wang.
\newblock {The shift between the Red Queen and the Red King effects in
  mutualisms.}
\newblock \emph{Scientific reports}, 5:\penalty0 8237, 2015.

\bibitem[Geerts and Pauw(2012)]{geerts:SAJB:2012}
S~Geerts and A~Pauw.
\newblock {The cost of being specialized: Pollinator limitation in the
  endangered geophyte Brunsvigia litoralis (Amaryllidaceae) in the Cape
  Floristic Region of South Africa}.
\newblock \emph{South African Journal of Botany}, 78:\penalty0 159--164, 2012.

\bibitem[Gokhale and Traulsen(2010)]{gokhale:PNAS:2010}
C.~S. Gokhale and A.~Traulsen.
\newblock Evolutionary games in the multiverse.
\newblock \emph{Proceedings of the National Academy of Sciences USA},
  107:\penalty0 5500--5504, 2010.

\bibitem[Gokhale and Traulsen(2012)]{gokhale:PRSB:2012}
C.~S. Gokhale and A.~Traulsen.
\newblock Mutualism and evolutionary multiplayer games: revisiting the {R}ed
  {K}ing.
\newblock \emph{Proceedings of the Royal Society B}, 279:\penalty0 4611--4616,
  2012.

\bibitem[Gokhale and Traulsen(2014)]{gokhale:DGAA:2014}
C.~S. Gokhale and A.~Traulsen.
\newblock Evolutionary multiplayer games.
\newblock \emph{Dynamic Games and Applications}, 4:\penalty0 468--488, 2014.

\bibitem[Hanski(2011)]{hanski:PNAS:2011}
I~A Hanski.
\newblock {Eco-evolutionary spatial dynamics in the Glanville fritillary
  butterfly.}
\newblock \emph{Proceedings of the National Academy of Sciences USA},
  108\penalty0 (35):\penalty0 14397--14404, 2011.

\bibitem[Hauert et~al.(2006{\natexlab{a}})Hauert, Holmes, and
  Doebeli]{hauert:PRSB:2006}
C.~Hauert, M.~Holmes, and M.~Doebeli.
\newblock Evolutionary games and population dynamics: maintenance of
  cooperation in public goods games.
\newblock \emph{Proceedings of the Royal Society B}, 273:\penalty0 2565--2570,
  2006{\natexlab{a}}.

\bibitem[Hauert et~al.(2006{\natexlab{b}})Hauert, Michor, Nowak, and
  Doebeli]{hauert:JTB:2006a}
C.~Hauert, F.~Michor, M.~A. Nowak, and M.~Doebeli.
\newblock Synergy and discounting of cooperation in social dilemmas.
\newblock \emph{Journal of Theoretical Biology}, 239:\penalty0 195--202,
  2006{\natexlab{b}}.

\bibitem[Hill and Pierce(1989)]{hill:OEC:1989}
C.~J. Hill and N.~E. Pierce.
\newblock The effect of adult diet on the biology of butterflies 1. the common
  imperial blue, \textit{Jalmenus evagoras}.
\newblock \emph{Oecologia}, 81:\penalty0 249--257, 1989.

\bibitem[Hinton(1951)]{hinton:PTENHS:1951}
H.~E. Hinton.
\newblock Myrmecophilous lycaenidae and other lepidoptera - a summary.
\newblock \emph{Proc. Trans. S. London Entomol. Nat. Hist. Soc.},
  1949-50:\penalty0 111--175, 1951.

\bibitem[Hoeksema and Kummel(2003)]{hoeksema:AmNat:2003}
J.~D. Hoeksema and M.~Kummel.
\newblock Ecological persistence of the plant-mycorrhizal mutualism: A
  hypothesis from species coexistence theory.
\newblock \emph{The American Naturalist}, 162:\penalty0 S40--S50, 2003.

\bibitem[Hofbauer(1996)]{hofbauer:JMB:1996}
J.~Hofbauer.
\newblock Evolutionary dynamics for bimatrix games: A {H}amiltonian system?
\newblock \emph{Journal of Mathematical Biology}, 34:\penalty0 675--688, 1996.

\bibitem[Hofbauer and Sigmund(1998)]{hofbauer:book:1998}
J.~Hofbauer and K.~Sigmund.
\newblock \emph{Evolutionary Games and Population Dynamics}.
\newblock Cambridge University Press, Cambridge, UK, 1998.

\bibitem[H{\"o}lldobler and Wilson(1990)]{hoelldobler:book:1990}
B.~H{\"o}lldobler and E.~O. Wilson.
\newblock \emph{The Ants}.
\newblock Belknap Press, 1990.

\bibitem[Janzen(1985)]{janzen:bookchapter:1985}
D.~H. Janzen.
\newblock The natural history of mutualisms.
\newblock In D.~H. Boucher, editor, \emph{The Biology of Mutualism}, pages
  1--28. Oxford University Press, New York, 1985.

\bibitem[Johnstone and Bshary(2002)]{johnstone:ECL:2002}
R.~A. Johnstone and R.~Bshary.
\newblock From parasitism to mutualism: partner control in asymmetric
  interactions.
\newblock \emph{Ecology Letters}, 5:\penalty0 634--639, 2002.

\bibitem[Kiers et~al.(2003)Kiers, Rousseau, West, and
  Denison]{kiers:Nature:2003}
E.~T. Kiers, R.~A. Rousseau, S.~A. West, and R.~F. Denison.
\newblock Host sanctions and the legume-rhizobium mutualism.
\newblock \emph{Nature}, 425:\penalty0 78--81, 2003.

\bibitem[Kiers et~al.(2011)Kiers, Duhamel, Beesetty, Mensah, Franken,
  Verbruggen, Fellbaum, Kowalchuk, Hart, Bago, Palmer, West, Vandenkoornhuyse,
  Jansa, and B{\"u}cking]{kiers:Science:2011}
E~Toby Kiers, Marie Duhamel, Yugandhar Beesetty, Jerry~A Mensah, Oscar Franken,
  Erik Verbruggen, Carl~R Fellbaum, Georg~A Kowalchuk, Miranda~M Hart, Alberto
  Bago, Todd~M Palmer, Stuart~A West, Philippe Vandenkoornhuyse, Jan Jansa, and
  Heike B{\"u}cking.
\newblock Reciprocal rewards stabilize cooperation in the mycorrhizal
  symbiosis.
\newblock \emph{Science}, 333:\penalty0 880--882, Jan 2011.

\bibitem[Maynard~Smith(1982)]{maynard-smith:book:1982}
J.~Maynard~Smith.
\newblock \emph{Evolution and the Theory of Games}.
\newblock Cambridge University Press, Cambridge, 1982.

\bibitem[McFall-Ngai(2014)]{mcfallngai:PLoSB:2014}
Margaret McFall-Ngai.
\newblock {Divining the Essence of Symbiosis: Insights from the Squid-Vibrio
  Model}.
\newblock \emph{PLoS Biology}, 12\penalty0 (2), 2014.

\bibitem[McGlynn(1999)]{mcglynn:JB:1999}
Terrence~P McGlynn.
\newblock {The Worldwide Transfer of Ants: Geographical Distribution and
  Ecological Invasions}.
\newblock \emph{Journal of Biogeography}, 26\penalty0 (3):\penalty0 535--548,
  1999.

\bibitem[Nazzi et~al.(2012)Nazzi, Brown, Annoscia, Del~Piccolo, Di~Prisco,
  Varricchio, Della~Vedova, Cattonaro, Caprio, and
  Pennacchio]{nazzi:PLosPath:2012}
Francesco Nazzi, Sam~P Brown, Desiderato Annoscia, Fabio Del~Piccolo, Gennaro
  Di~Prisco, Paola Varricchio, Giorgio Della~Vedova, Federica Cattonaro, Emilio
  Caprio, and Francesco Pennacchio.
\newblock {Synergistic Parasite-Pathogen Interactions Mediated by Host Immunity
  Can Drive the Collapse of Honeybee Colonies}.
\newblock \emph{PLoS Pathogens}, 8\penalty0 (6):\penalty0 e1002735, 2012.

\bibitem[No{\"e}(2001)]{noe:book:2001}
R.~No{\"e}.
\newblock Biological markets: partner choice as the driving force behind the
  evolution of mutualisms.
\newblock In Ronald No{\"e}, Jan~A.R.A.M. van Hooff, and Peter Hammerstein,
  editors, \emph{Economics in Nature: Social Dilemmas, Mate Choice and
  Biological Markets}. Cambridge University Press, 2001.

\bibitem[No{\"e} and Hammerstein(1995)]{noe:TREE:1995}
Ronald No{\"e} and Peter Hammerstein.
\newblock {Biological markets}.
\newblock \emph{Trends in Ecology and Evolution}, 10\penalty0 (8):\penalty0
  336--339, August 1995.

\bibitem[Nowak(2006)]{nowak:book:2006}
M.~A. Nowak.
\newblock \emph{Evolutionary Dynamics}.
\newblock Harvard University Press, Cambridge MA, 2006.

\bibitem[Nowak and Sigmund(2004)]{nowak:Science:2004}
M.~A. Nowak and K.~Sigmund.
\newblock Evolutionary dynamics of biological games.
\newblock \emph{Science}, 303:\penalty0 793--799, 2004.

\bibitem[Nyholm and McFall-Ngai(2004)]{nyholm:NRM:2004}
Spencer~V Nyholm and Margaret McFall-Ngai.
\newblock {The winnowing: establishing the squid|[ndash]|vibrio symbiosis}.
\newblock \emph{Nature Reviews Microbiology}, 2\penalty0 (8):\penalty0
  632--642, 2004.

\bibitem[Pacheco et~al.(2009)Pacheco, Santos, Souza, and
  Skyrms]{pacheco:PRSB:2009}
J.~M. Pacheco, F.~C. Santos, M.~O. Souza, and B.~Skyrms.
\newblock Evolutionary dynamics of collective action in n-person stag hunt
  dilemmas.
\newblock \emph{Proceedings of the Royal Society B}, 276:\penalty0 315--321,
  2009.

\bibitem[Pierce et~al.(2002)Pierce, Braby, Heath, Lohman, Mathew, Rand, and
  Travassos]{pierce:ARE:2002}
N.~E. Pierce, M.~F. Braby, A.~Heath, D.~J. Lohman, J.~Mathew, D.~B. Rand, and
  M.~A. Travassos.
\newblock {The Ecology and Evolution of Ant Association in the Lycaenidae
  (Lepidoptera)}.
\newblock \emph{Annual Review of Entomology}, 47:\penalty0 733--770, 2002.

\bibitem[Pierce et~al.(1987)Pierce, Kitching, Buckley, Taylor, and
  Benbow]{pierce:BES:1987}
Naomi~E Pierce, R~L Kitching, R~C Buckley, M~F~J Taylor, and K~F Benbow.
\newblock The costs and benefits of cooperation between the australian lycaenid
  butterfly, \textit{Jalmenus evagoras}, and its attendant ants.
\newblock \emph{Behavioral Ecology and Sociobiology}, 21:\penalty0 237--248,
  Jun 1987.

\bibitem[Post and Palkovacs(2009)]{post:PTRSB:2009}
D~M Post and E~P Palkovacs.
\newblock {Eco-evolutionary feedbacks in community and ecosystem ecology:
  interactions between the ecological theatre and the evolutionary play}.
\newblock \emph{Philosophical Transactions of the Royal Society B: Biological
  Sciences}, 364\penalty0 (1523):\penalty0 1629--1640, 2009.

\bibitem[Poulin and Vickery(1995)]{poulin:JTB:1995}
R.~Poulin and W.~L. Vickery.
\newblock Cleaning symbiosis as an evolutionary game: to cheat or not to cheat?
\newblock \emph{Journal of Theoretical Biology}, 175:\penalty0 63--70, 1995.

\bibitem[Rafferty et~al.(2015)Rafferty, CaraDonna, and
  Bronstein]{rafferty:Oikos:2015}
Nicole~E Rafferty, Paul~J CaraDonna, and Judith~L Bronstein.
\newblock {Phenological shifts and the fate of mutualisms.}
\newblock \emph{Oikos}, 124\penalty0 (1):\penalty0 14--21, 2015.

\bibitem[Richardson et~al.(2000)Richardson, Allsopp, D'Antonio, Milton, and
  Rejm{\'a}nek]{richardson:BR:2000}
D~M Richardson, N~Allsopp, C~M D'Antonio, S~J Milton, and M~Rejm{\'a}nek.
\newblock {Plant invasions--the role of mutualisms.}
\newblock \emph{Biological Reviews}, 75\penalty0 (1):\penalty0 65--93, 2000.

\bibitem[Roughgarden(1976)]{roughgarden:TPB:1976}
J.~Roughgarden.
\newblock Resource partitioning among competing species - a coevolutionary
  approach.
\newblock \emph{Theoretical Population Biology}, 9:\penalty0 288--424, 1976.

\bibitem[Roughgarden et~al.(1983)Roughgarden, Heckel, and
  Fuentes]{roughgarden:book:1983}
J.~Roughgarden, D.~Heckel, and E.~Fuentes.
\newblock \emph{Lizard Ecology: Studies of a Model Organism}, chapter
  Coevolutionary theory and the biogeography and community structure of
  {\it{Anolis}}., pages 371--410.
\newblock Harvard University Press, 1983.

\bibitem[Sanchez and Gore(2013)]{sanchez:PLoSB:2013}
A~Sanchez and J~Gore.
\newblock {Feedback between Population and Evolutionary Dynamics Determines the
  Fate of Social Microbial Populations}.
\newblock \emph{PLoS Biology}, 11\penalty0 (4):\penalty0 e1001547, 2013.

\bibitem[Schluter and Foster(2012)]{schluter:PlosB:2012}
Jonas Schluter and Kevin~R Foster.
\newblock The evolution of mutualism in gut microbiota via host epithelial
  selection.
\newblock \emph{PLoS Biology}, 10\penalty0 (11):\penalty0 e1001424, 2012.

\bibitem[Souza et~al.(2009)Souza, Pacheco, and Santos]{souza:JTB:2009}
M.~O. Souza, J.~M. Pacheco, and F.~C. Santos.
\newblock Evolution of cooperation under n-person snowdrift games.
\newblock \emph{Journal of Theoretical Biology}, 260:\penalty0 581--588, 2009.

\bibitem[Stadler and Dixon(2008)]{stadler:book:2008}
B.~Stadler and A.~F.~G. Dixon.
\newblock \emph{Mutualism: Ants and their Insect partners}.
\newblock Cambridge University Press, 2008.

\bibitem[Stanton(2003)]{stanton:AmNat:2003}
Maureen~L Stanton.
\newblock {Interacting Guilds: Moving beyond the Pairwise Perspective on
  Mutualisms}.
\newblock \emph{The American Naturalist}, 162\penalty0 (s4):\penalty0 S10--S23,
  2003.

\bibitem[Trivers(1971)]{trivers:QRB:1971}
R.~L. Trivers.
\newblock The evolution of reciprocal altruism.
\newblock \emph{The Quarterly Review of Biology}, 46:\penalty0 35--57, 1971.

\bibitem[Warren and Bradford(2014)]{warren:GCB:2014}
Robert J~II Warren and Mark~A Bradford.
\newblock {Mutualism fails when climate response differs between interacting
  species}.
\newblock \emph{Global Change Biology}, 20\penalty0 (2):\penalty0 466--474,
  2014.

\bibitem[Weibull(1995)]{weibull:book:1995}
J.~W. Weibull.
\newblock \emph{Evolutionary Game Theory}.
\newblock MIT Press, Cambridge, 1995.

\bibitem[Wilson(1983)]{wilson:AmNat:1983}
D.~S. Wilson.
\newblock The effect of population structure on the evolution of mutualism: A
  field test involving burying beetles and their phoretic mites.
\newblock \emph{The American Naturalist}, 121\penalty0 (6):\penalty0 851--870,
  1983.

\bibitem[Yao and Akimoto(2009)]{yao:JIS:2009}
Izumi Yao and Shin-Ichi Akimoto.
\newblock {Seasonal changes in the genetic structure of an aphid-ant mutualism
  as revealed using microsatellite analysis of the aphid Tuberculatus
  quercicola and the ant Formica yessensis}.
\newblock \emph{Journal of Insect Science}, 9\penalty0 (1):\penalty0 9--18,
  2009.

\bibitem[Yao et~al.(2000)Yao, Shibao, and Akimoto]{yao:Oikos:2000}
Izumi Yao, Harunobu Shibao, and Shin-Ichi Akimoto.
\newblock {Costs and Benefits of Ant Attendance to the Drepanosiphid Aphid
  Tuberculatus quercicola}.
\newblock \emph{Oikos}, 89\penalty0 (1):\penalty0 3--10, 2000.

\end{thebibliography}


\renewcommand{\theequation}{A.\arabic{equation}}
\setcounter{equation}{0}

\renewcommand{\thefigure}{A.\arabic{figure}}
\setcounter{figure}{0}

\begin{appendices}

\section{Interspecies Evolutionary Dynamics}

Traditional coevolutionary models consider interspecific dependence only \citep{roughgarden:TPB:1976,roughgarden:book:1983}.
Since in our case each the interactions between the species are mutualistic and each species consists of two types of individuals ``Generous" and ``Selfish", the following Snowdrift game is an appropriate representation of the interactions.
%This is because we have neglected intraspecific interactions as mentioned earlier.

%For different types of interactions between species different models need to be defined \citep{poulin:JTB:1995,doebeli:PNAS:1998,noe:book:2001,johnstone:ECL:2002,bergstrom:PNAS:2003,hoeksema:AmNat:2003,akcay:PRSB:2007,bshary:Nature:2008}.




\subsection*{The snowdrift game}
\label{appA}
\subsubsection*{Two player setting}
%So far, we have described general games within and between species, now we turn to a particular game which of interest to us when considering mutualism.
Two drivers are stuck in a snowdrift.
They must shovel away the snow (paying the cost $c$) to reach home (benefit $b$) but there are three possible outcomes to this scenario.
One of the driver shovels while the other stays warm in the care ($b-c$ and $b$), both the drivers share the workload and shovel away the snow ($b-c/2$ and $b-c/2$) or none of them gets out of the care and they both remain stuck ($0$ and $0$).

Putting this game in perspective of the two species (i.e. the two drivers represent the two different species) we get the matrix,\\
%
\begin{equation}\label{}
\begin{array}{cc|cc}
\multicolumn{4}{l}{\textit{Species 1 payoff:}} \\
\hline\hline
& & \multicolumn{2}{c}{\text{Species 2}}\\
&	&	G_2		&	S_2	\\
\hline
 \multirow{2}{*}{Species 1} & G_1 	& b-c/2 &	b-c \\
&	S_1	&  b & 0 \\
 \hline\hline
\end{array}
\hspace{1cm}
\begin{array}{cc|cc}
\multicolumn{4}{l}{\textit{Species 2 payoff:}} \\
\hline\hline
& & \multicolumn{2}{c}{\text{Species 1}}\\
&	&	G_1		&	S_1	\\
\hline
 \multirow{2}{*}{Species 2} & G_2 	& b-c/2 &	b-c \\
&	S_2	& b & 0 \nonumber \\
 \hline\hline
\end{array}
\end{equation}
%
where strategy $G$ stands for being \textit{``Generous"} and shoveling the snow while $S$ stands for being \textit{``Selfish"} and just sitting in the car.
For $b=2$ and $c=1$ we recover the matrix used in \citep{bergstrom:PNAS:2003}.

For the snowdrift game in a single population for which the pairings are formed at random, there exists a single, stable internal equilibrium.
Hence the population will evolve to a polymorphism which is a combination of \textit{``Generous"} and \textit{``Selfish"} individuals.
But in a two species system (pairs still random, but one from each species), this stable equilibrium turns into a saddle point: a small deviation from this fixed point leads the system to one of the stable fixed point where one of the species is completely \textit{``Generous"} and the other one is completely \textit{``Selfish"}.

\subsection*{Multiplayer setting}
\label{appB}

Following Souza et al. \citep{souza:JTB:2009},  
a multiplayer snowdrift game can be described by the payoff entries
\begin{eqnarray}
\Pi_{G_1} (k)  &=& \begin{cases} b-\frac{c}{k} & \textrm{if } k \geq M \\  -\frac{c}{M} & \textrm{if } k < M \end{cases} \nonumber
\\
\Pi_{S_1} (k)  &=& \begin{cases} b & \textrm{if } k \geq M \\ 0 & \textrm{if } k < M. \end{cases}
\label{eqintergamepayoffs}
\end{eqnarray}
%
All players get the benefit $b$ if the number of generous individuals in both species combined, $k$, is greater than or equal to the threshold $M$.
For the generous individuals, their effort is subtracted from the payoffs.
The effort is shared if the quorum size is met ($\frac{c}{M}$), but is in vain for $k<M$. \marcus{I'm confused here: why is $\frac{c}{k}$ lost if above the threshold but  $\frac{c}{M}$ lost if not?}
\cha{So below the threshold all cooperators are trying their best by putting in $c/M$ as $M$ is the threshold but as soon as the threshold is crossed then they can put in less $c/k$ as $k$ will be larger than $M$}
For two player games we had $k=1$ but multiplayer games provide the possibility of exploring this threshold aspect of collective action games.
From these payoff entries we need to calculate the average fitnesses.
For simplicity we just illustrate the fitnesses of the strategies in species $1$.
For a $d_1^{inter}$ player game for species $1$ we need to pick $d_1^{inter}-1$ other individuals from species $2$.
Assuming random sampling the composition of the formed groups is given by a binomial distribution.
Summing over all possible compositions of groups we arrive at  the average fitnesses of the two strategies in species $1$,
%
\begin{align}
f^{inter}_{G_1} (y) &= \sum_{k=0}^{d_1^{inter} -1} \binom{d_1^{inter} -1}{k}y^k (1-y)^{d_1^{inter} -1-k} \Pi_{G_1}(k+1) \nonumber \\
f^{inter}_{S_1} (y) &= \sum_{k=0}^{d_1^{inter} -1} \binom{d_1^{inter} -1}{k}y^k (1-y)^{d_1^{inter} -1-k} \Pi_{S_1}(k),
\label{interfiteqs}
\end{align}
%
and similarly $f_{G_2}^inter$ and $f_{S_2}^inter$ for species $2$.

Note that here for the sake of notation we have assumed the same values of benefits and costs, thresholds for the two species. However along with the number of player $d_1^{inter}$ and $d_2^{inter}$, these parameters could be very well different for the two species.
For asymmetric bi-matrix games there is a difference in the dynamics between the standard replicator dynamics and the alternative dynamics put forward by Maynard-Smith \citep{maynard-smith:book:1982}.
In this case the replicator equations cannot be simplified by removing the average fitness from the denominator and can give rise to qualitatively different dynamics. 
Then one has to resort to difference rather than differential equations.

\section{Intraspecies Evolutionary Dynamics}
\label{appB}

For elucidating the intraspecies dynamics we will focus on species $1$ as the analysis is analogous for species $2$.
Within species dynamics can in principle be completely different from the between species interactions. 
We can have a multistrategy multiplayer game within a species but to keep things simple we assume a two strategy multiplayer game.
The partitioning of the individuals into two strategies follows the same partitioning as in the inter species interactions as of ``Generous" and ``Selfish". 
In principle we could have two different labels for the strategies in the intraspecies interactions and the ``Generous" and ``Selfish" categories could be split into them.
However for the sake of simplicity we assume the same categorisation as at the inter species level.
%However we can relabel them as ``Cooperators" and ``Defector" for the sake of the interactions structure which we will be making use of.
%Note that the ``Generous" in the interspecies interactions need not always be the ``Cooperators" for the within species interaction but again for the sake of simplicity we will assume it  to be so. \marcus{Ah! now I get it. I guess we need to highlight this earlier on, as it's a strong condition: I found myself wondering whether Generous in inter $\leftrightarrow$ Cooperator in intra, or not...}

\subsection*{Synergy/Discounting Framework}
We model the within species interactions by making use of a general framework of costs and non-linear benefits \citep{eshel:AmNat:1988,hauert:JTB:2006a} which can potentially encompass all different types of (traditionally studied) social interaction structures qualitatively \citep{nowak:book:2006}, i.e., dominance of either type, coexistence and bistability.
Since the categorisation of the strategies at the intraspecies level is the same as that of the inter species level, for species $1$, $x$ and $1-x$, are the frequencies of ``Generous" and ``Selfish" type. \marcus{Q: is this because they are the very same players? i.e. are we assuming a Generous player in the inter is a Cooperative one in the intra?}
\cha{Yes. Now described above}
The ``Generous" and ``Selfish" in species $1$ play a $d_1^{intra}$ player game.
Thus the fitnesses of of the two types are defined as \citep{hauert:JTB:2006a},
%
\begin{align}
	f^{intra}_{G_1} (x) &= \sum_{k=0}^{d_1^{intra} -1} \binom{d_1^{intra} -1}{k}x^k (1-x)^{d_1^{intra} -1-k} \Gamma_{G_1}(k+1) \nonumber \\
	f^{intra}_{S_1} (x) &= \sum_{k=0}^{d_1^{intra} -1} \binom{d_1^{intra} -1}{k}x^k (1-x)^{d_1^{intra} -1-k} \Gamma_{S_1}(k).
\label{intrafiteqs}
\end{align}
%
where the payoffs are given by,
\begin{align}
	\Gamma_{S_1} (k) = \frac{\tilde{b}}{d_1^{intra}} \sum_{i=0}^{k-1} \omega^i \nonumber \\
	\Gamma_{G_1} (k) = \Gamma_{S_1} (k) - \tilde{c}.
\label{eqintragamepayoffs}
\end{align}
%
Thus the ``Selfish" get a fraction of the benefit which is scaled by the factor $\omega$, which determines whether the benefits are linearly accumulating ($\omega=1$) for increasing number of ``Generous" individuals, synergistically enhanced ($\omega>1$) or saturating ($\omega<1$).
Note that the costs and benefits in the within species game need not be the same as in between species ($b\neq \tilde{b}$ and $c \neq \tilde{c}$).


\section{Combined Evolutionary Dynamics}
\label{app:combineddyn}

The average payoffs are then assumed to be a linear combination of the interspecies and intraspecies interactions where the parameter $p$ determines the strength of each of the interactions such that,
%
\begin{align}
	f_{G_1} (x,y) &= p f^{inter}_{G_1} (y) + (1-p) f^{intra}_{G_1} (x) \nonumber \\
	f_{S_1} (x,y) &= p f^{inter}_{S_1} (y) + (1-p) f^{intra}_{S_1} (x).
\label{fiteqs}
\end{align}
%
Following the same procedure for the two strategies in species $2$ leads to the average fitness
%
\begin{align}
\bar{f}_1 (x,y) &= x f_{G_1} (x,y)+(1-x) f_ {S_1}(x,y) \nonumber \\
\bar{f}_2 (x,y) &= y f_{G_2} (x,y)+(1-y) f_{S_2}(x,y).
\label{avgfiteqs}
\end{align}
%
The time evolution of the ``Generous" types in both the species will give us the complete dynamics of the system.
However since the two interaction species are by definition different organisms, they can have different rates of evolution.
Thus if species 1 evolves at the rate $r_x$ while species 2 at rate $r_y$ then we have,
\begin{align}
\dot{x} &= r_x x \left(f_{G_1}(x,y) -  \bar{f}_1(x,y) \right) \nonumber \\
\dot{y} &= r_y y \left(f_{G_2}(x,y) -  \bar{f}_2(x,y) \right).
\label{eq:repeqapp}
\end{align}


\begin{figure}[h]
\begin{center}
\includegraphics[width=\columnwidth]{../Figures/Dynamicsacrossp_reduced.pdf}
\caption{
$d_1^{inter} = d_2^{inter} = 5$, $b = 2$, $r_x = r_y/8$, $M_1 = M_2 = 1$ and $c=1$ for the interspecies game. As for the intraspecies games we have $d_1^{intra} = d_2^{intra} = 5$ and $\tilde{b} = 10 $ with 
(a) $\tilde{c} = 3$, $\omega = 3/4$, 
(b) $\tilde{c} = 1$, $\omega = 3/4$, 
(c) $\tilde{c} = 1$, $\omega = 4/3$ and 
(d) $\tilde{c} = 3$, $\omega = 4/3$, the exact same parameter values as in \citep{hauert:JTB:2006a}.
\label{fig:appendix}
}
\end{center}
\end{figure}


%\cha{\section*{Asymmetries}}
%
%\cha{This between and within species model is a powerful way of introducing a lot of variability into the dynamics,
%\begin{align}
%	d_1 &\neq d_2 \\
%	d^{inter} &\neq d^{intra} \\
%	M_1 &\neq M_2 \\
%	b &\neq \tilde{b} \\
%	c &\neq \tilde{c} \\
%	r_x &\neq r_y \\
%	&\vdots
%\end{align}
%and various combinations of these. We should justify why we don't do this here and why we do vary the ones that we do.}


%\subsection{Dynamics in asymmetric conditions}
%
%%We have addressed two kinds of asymmetries in the game, the number of player and the thresholds in the two species.
%%We denote the number of players for species $1$ and species $2$ as $d_1$ and $d_2$, respectively, as in Fig.\ \ref{fig:counter}.
%%That is if species $2$ is playing a $d_2$ player game it means that one player from species $2$ interacts with $d_2-1$ players of species $1$.
%%For an asymmetry in the thresholds we use the two parameters $M_1\geq1$ and $M_2\geq1$ for the two species, respectively.
%
%For asymmetric bimatrix games, there is a difference in the dynamics between the standard replicator dynamics and the 
%alternative dynamics put forward by Maynard-Smith \citep{maynard-smith:1982to}.
%For this dynamics, the average fitness of each species appears as a denominator,
%\begin{align}
%\dot{x} &= r_x x \left(f_{G_1}(y) -  \bar{f}_1(x,y) \right)/\bar{f}_1(x,y) \nonumber \\
%\dot{y} &= r_y y \left(f_{G_2}(x) -  \bar{f}_2(x,y) \right)/\bar{f}_2(x,y).
%\label{eq:repeqs}
%\end{align}
%In our asymmetric bimatrix game, the fixed point stability is affected by the choice of the dynamics, in contrast to the case of symmetric games. 
%%In Fig.\ \ref{fig:thresholdsmodrep}, we illustrate that the dynamics is different between the usual replicator dynamics and Eqs. \ref{eq:repeqs}
%
%For $d_1=d_2 \geq 5$, the exact coordinates of the fixed point must be computed numerically \citep{abel:AO:1824,stewart:book:2004}.


\section{Population dynamics}

For brevity we begin with the description of population dynamics in species 1.
The two types in species 1, ``Generous" and ``Selfish" need not sum up to $1$ i.e. the population may not always be at its carrying capacity.
Hence if the empty space in the niche occupied by species $1$ is $z_1$, then we have $x_1 + x_2 + z_1 = $ where $x_1$ and $x_2$ are the densities of ``Generous" and ``Selfish" types.
The population dynamics then is dictated by,
%
\begin{align}
	\dot{x_1} &= r_x x_1 (z_1 f_{G_1} - e_1) \nonumber \\
	\dot{x_2} &= r_x x_2 (z_1 f_{S_1} - e_1) \\
	\dot{z_1} &= - \dot{x_1} - \dot{x_2} \nonumber
\end{align}
%
and for species 2
\begin{align}
	\dot{y_1} &= r_y y_1 (z_2 f_{G_2} - e_2) \nonumber \\
	\dot{y_2} &= r_y y_2 (z_2 f_{S_2} - e_2) \\
	\dot{z_2} &= - \dot{y_1} - \dot{y_2} \nonumber
\end{align}
%
We have $e_1$ and $e_2$ as the death rates for the two species. 
For the special case of  $e_1 = \frac{z_1 (x_1 f_{x_1} + x_2 f_{x_2}) }{x_1 + x_2}$ and $e_2 = \frac{z_2 (y_1 f_{G_2} + y_2 f_{S_2}) }{y_1 + y_2}$ we recover the two species replicator dynamics as in Eqs.~\ref{eq:repeqapp}. 
The fitnesses however need to be reevaluated in this setup.
For example in species 1 the fitness for type $G_1$ is,
%
\begin{align}
	f_{G_1}^{inter} &= \sum_{S=2}^{d_1} \binom{d_1 -1}{S-1} z_2 ^{d_1 -S} (1-z_2)^{S-1} P_G^{inter}(S,y_1,y_2,z_2) \nonumber \\
	f_{G_1}^{intra} &= \sum_{S=2}^{d_1} \binom{d_1 -1}{S-1} z_1 ^{d_1 -S} (1-z_1)^{S-1} P_G^{intra}(S,x_1,x_2,z_1) \\
	f_{G_1} &= f_{G_1}^{inter} + f_{G_1}^{intra}
\end{align}
%
and similarly for type $S_1$ where the payoff functions are defined as,
%
\begin{align}
	P_G^{inter}(S,p,q,r) &= \sum_{k=0}^{S-1} V(S,p,q,r) \Pi_{G_1}(k+1) \\
	P_G^{intra}(S,p,q,r) &= \sum_{k=0}^{S-1} V(S,p,q,r) \Gamma_{G_1}(k+1) \\
	P_S^{inter}(S,p,q,r) &= \sum_{k=0}^{S-1} V(S,p,q,r) \Pi_{S_1}(k) \\
	P_S^{intra}(S,p,q,r) &= \sum_{k=0}^{S-1} V(S,p,q,r) \Gamma_{S_1}(k)
\end{align}
%
where $V(S,p,q,r) = \binom{S-1}{k} \left( \frac{p}{1-r}\right)^k  \left(\frac{q}{1-r}\right)^{S-1-k}$ is the probability of having a $k$ ``Generous"(Cooperator) individuals and $S-1-k$ ``Selfish"(Defector) individuals in the inter(intra) species game.
and the actual payoffs are calculated as per Eqs.~\ref{eqintergamepayoffs} and \ref{eqintragamepayoffs}.

\end{appendices}

\end{document}